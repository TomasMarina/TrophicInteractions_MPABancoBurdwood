\documentclass[preprint, 3p,
authoryear]{elsarticle} %review=doublespace preprint=single 5p=2 column
%%% Begin My package additions %%%%%%%%%%%%%%%%%%%

\usepackage[hyphens]{url}

  \journal{MEPS, Oikos, Ecological Applications, J of Applied Ecology
(?)} % Sets Journal name

\usepackage{graphicx}
%%%%%%%%%%%%%%%% end my additions to header

\usepackage[T1]{fontenc}
\usepackage{lmodern}
\usepackage{amssymb,amsmath}
% TODO: Currently lineno needs to be loaded after amsmath because of conflict
% https://github.com/latex-lineno/lineno/issues/5
\usepackage{lineno} % add
  \linenumbers % turns line numbering on
\usepackage{ifxetex,ifluatex}
\usepackage{fixltx2e} % provides \textsubscript
% use upquote if available, for straight quotes in verbatim environments
\IfFileExists{upquote.sty}{\usepackage{upquote}}{}
\ifnum 0\ifxetex 1\fi\ifluatex 1\fi=0 % if pdftex
  \usepackage[utf8]{inputenc}
\else % if luatex or xelatex
  \usepackage{fontspec}
  \ifxetex
    \usepackage{xltxtra,xunicode}
  \fi
  \defaultfontfeatures{Mapping=tex-text,Scale=MatchLowercase}
  \newcommand{\euro}{€}
\fi
% use microtype if available
\IfFileExists{microtype.sty}{\usepackage{microtype}}{}

\ifxetex
  \usepackage[setpagesize=false, % page size defined by xetex
              unicode=false, % unicode breaks when used with xetex
              xetex]{hyperref}
\else
  \usepackage[unicode=true]{hyperref}
\fi
\hypersetup{breaklinks=true,
            bookmarks=true,
            pdfauthor={},
            pdftitle={The complex network of trophic interactions in a subAntarctic oceanic Marine Protected Area},
            colorlinks=false,
            urlcolor=blue,
            linkcolor=magenta,
            pdfborder={0 0 0}}

\setcounter{secnumdepth}{5}
% Pandoc toggle for numbering sections (defaults to be off)


% tightlist command for lists without linebreak
\providecommand{\tightlist}{%
  \setlength{\itemsep}{0pt}\setlength{\parskip}{0pt}}

% From pandoc table feature
\usepackage{longtable,booktabs,array}
\usepackage{calc} % for calculating minipage widths
% Correct order of tables after \paragraph or \subparagraph
\usepackage{etoolbox}
\makeatletter
\patchcmd\longtable{\par}{\if@noskipsec\mbox{}\fi\par}{}{}
\makeatother
% Allow footnotes in longtable head/foot
\IfFileExists{footnotehyper.sty}{\usepackage{footnotehyper}}{\usepackage{footnote}}
\makesavenoteenv{longtable}

% Pandoc citation processing
\newlength{\cslhangindent}
\setlength{\cslhangindent}{1.5em}
\newlength{\csllabelwidth}
\setlength{\csllabelwidth}{3em}
\newlength{\cslentryspacingunit} % times entry-spacing
\setlength{\cslentryspacingunit}{\parskip}
% for Pandoc 2.8 to 2.10.1
\newenvironment{cslreferences}%
  {}%
  {\par}
% For Pandoc 2.11+
\newenvironment{CSLReferences}[2] % #1 hanging-ident, #2 entry spacing
 {% don't indent paragraphs
  \setlength{\parindent}{0pt}
  % turn on hanging indent if param 1 is 1
  \ifodd #1
  \let\oldpar\par
  \def\par{\hangindent=\cslhangindent\oldpar}
  \fi
  % set entry spacing
  \setlength{\parskip}{#2\cslentryspacingunit}
 }%
 {}
\usepackage{calc}
\newcommand{\CSLBlock}[1]{#1\hfill\break}
\newcommand{\CSLLeftMargin}[1]{\parbox[t]{\csllabelwidth}{#1}}
\newcommand{\CSLRightInline}[1]{\parbox[t]{\linewidth - \csllabelwidth}{#1}\break}
\newcommand{\CSLIndent}[1]{\hspace{\cslhangindent}#1}


\usepackage{booktabs}
\usepackage{longtable}
\usepackage{array}
\usepackage{multirow}
\usepackage{wrapfig}
\usepackage{float}
\usepackage{colortbl}
\usepackage{pdflscape}
\usepackage{tabu}
\usepackage{threeparttable}
\usepackage{threeparttablex}
\usepackage[normalem]{ulem}
\usepackage{makecell}
\usepackage{xcolor}



\begin{document}


\begin{frontmatter}

  \title{The complex network of trophic interactions in a subAntarctic
oceanic Marine Protected Area}
    \author[]{Tomás I. Marina\textsuperscript{a}%
  %
  }
   \ead{tomasimarina@gmail.com} 
    \author[]{Irene R. Schloss\textsuperscript{a,b,c}%
  %
  }
  
    \author[]{Gustavo A. Lovrich\textsuperscript{a}%
  %
  }
  
    \author[]{Claudia C. Boy\textsuperscript{a}%
  %
  }
  
    \author[]{Daniel O. Bruno\textsuperscript{a,c}%
  %
  }
  
    \author[]{Fabiana L. Capitanio\textsuperscript{d,e}%
  %
  }
  
    \author[]{Sergio M. Delpiani\textsuperscript{f}%
  %
  }
  
    \author[]{Juan Martín Díaz de Astarloa\textsuperscript{e}%
  %
  }
  
    \author[]{Cintia Fraysse\textsuperscript{a}%
  %
  }
  
    \author[]{Virginia A. García Alonso\textsuperscript{d,e}%
  %
  }
  
    \author[]{Andrea Raya Rey\textsuperscript{a,c,g}%
  %
  }
  
    \author[]{Laura Schejter\textsuperscript{h}%
  %
  }
  
    \author[]{Mariela L. Spinelli\textsuperscript{d,e}%
  %
  }
  
    \author[]{Marcos Tatián\textsuperscript{i,j}%
  %
  }
  
    \author[]{Diego Urteaga\textsuperscript{k}%
  %
  }
  
    \author[]{Luciana Riccialdelli\textsuperscript{a}%
  %
  }
  
      \affiliation[1]{Centro Austral de Investigaciones Cientificas
(CADIC-CONICET), Argentina}
    \affiliation[2]{Instituto Antartico Argentino (IAA), Argentina}
    \affiliation[3]{Instituto de Ciencias Polares, Ambiente y Recursos
Naturales (ICPA), Universidad de Tierra del Fuego (UNTDF), Argentina}
    \affiliation[4]{Facultad de Ciencias Exactas y Naturales,
Universidad de Buenos Aires (UBA), Argentina}
    \affiliation[5]{Instituto de Biodiversidad y Biologia Experimental y
Aplicada (IBBEA), Universidad de Buenos Aires-CONICET, Argentina}
    \affiliation[6]{Instituto de Investigaciones Marinas y Costeras
(IIMYC), Universidad Nacional de Mar del Plata-CONICET, Argentina}
    \affiliation[7]{Wildlife Conservation Society, Argentina}
    \affiliation[8]{Instituto Nacional de Investigacion y Desarrollo
Pesquero (INIDEP), Consejo Nacional de Investigaciones Cientificas y
Tecnicas (CONICET), Argentina}
    \affiliation[9]{Facultad de Ciencias Exactas, Fisicas y Naturales,
Universidad Nacional de Cordoba (UNC), Argentina}
    \affiliation[10]{Instituto de Diversidad y Ecologia Animal
(IDEA-CONICET)}
    \affiliation[11]{Museo Argentino de Ciencias Naturales ``Bernardino
Rivadavia'', Argentina}
    \cortext[cor1]{Corresponding author}
  
  \begin{abstract}
  Globally, the total area of the world ocean designated under marine
  protection has increased in the recent decades. The majority of these
  Marine Protected Areas (MPAs) focus on the presence of particularly
  vulnerable, keystone, or charismatic species, large numbers of endemic
  species, and/or high biodiversity across taxonomic levels. In the
  sub-Antarctic region, the level of ocean protection is mainly
  associated to oceanic islands, except for the MPAs Namuncurá -
  Burdwood Bank I and II (MPA N-BB, \textasciitilde53º--55ºS and
  \textasciitilde56º--62ºW), which are associated to a submarine plateau
  and its southern adjacent deep slope region, respectively. Here, we
  present the first detailed analysis of the network of predator-prey
  interactions (food web) for the MPA N-BB ecosystem. We applied a
  network approach to characterise the food web in terms of complexity
  and structure, and identifying the species' role in such a framework.
  In terms of complexity, the MPA N-BB food web consisted of 1788
  predator-prey interactions and 379 species, with a link density of
  4.72 and a connectance of 0.01. In terms of structure, almost half of
  the consumers were omnivores (0.48), and the network displayed a
  small-world pattern. These findings suggest that the ecosystem might
  be vulnerable to external perturbations targeting highly connected
  species, although other properties might provide resilience and
  resistance, resulting in a rearranged structure that preserves its
  original functions. Furthermore, we identified several species as
  important in terms of different aspects of trophic structure and
  functioning, and response to perturbations. We suggest that generalist
  species, mainly fishes, play a crucial role in the ecosystem's
  bentho-pelagic coupling process and that other species, besides the
  longtail southern cod \emph{Patagonotothen ramsayi} and the Fueguian
  sprat \emph{Sprattus fuegensis}, should be considered as relevant
  energy transfers for the ecosystem. Finally, we argue that the
  diversity of species, including both the benthic and pelagic habitats,
  is responsible for seccuring the connectivity within the food web
  against perturbations, therefore contributing to the structure and
  stability of the ecosystem.
  \end{abstract}
    \begin{keyword}
    Food web \sep Complexity \sep Structure \sep Burdwood Bank \sep 
    Southwest Atlantic
  \end{keyword}
  
 \end{frontmatter}

\hypertarget{introduction}{%
\section{Introduction}\label{introduction}}

The evidence of benefits provided by Marine Protected Areas (MPAs) as
well as the urgent need for ocean protection have driven an
unprecedented increase in the number of MPAs worldwide in recent years
(Roberts et al., 2017; Sala et al., 2018). Globally, the total area of
the world ocean designated under marine protection adds up to
approximately 29,600,000 km\textsuperscript{2}, distributed across
nearly 18,444 MPAs and covering 8.16\% of the ocean's surface (IUCN \&
UNEP-WCMC, 2023), and therefore approaching the 10\% goal of the
Convention of Biological Diversity (Secretariat of the convention on
biological diversity, 2004). Despite this progress, recent reports have
shown that actual protection has been overestimated because it includes
areas that are not yet effectively protected (only declared) as well as
areas that allow significant extractive activities (Sala et al., 2018).

In the sub-Antarctic region, the level of ocean protection is mainly
associated to oceanic islands, such as the South Georgias and South
Sandwich, Bouvet, Prince Edward, and Macquarie islands (IUCN \&
UNEP-WCMC, 2023). Interestingly, the case of the MPAs Namuncurá -
Burdwood Bank I and II (MPA N-BB, \textasciitilde53º--55ºS
\textasciitilde56º--62ºW), which is the focus of this work, is unique
since these MPAs are associated to a submarine plateau and its southern
adjacent deep slope region, respectively (Falabella, 2017; Schejter et
al., 2020). In addition, such MPAs are part of a network of protected
areas in the sub-Antarctic area (jointly with MPA Yaganes) that aims to
protect this southern region in order to contribute to global ocean
health.

Many of these MPAs focus on the presence of particularly vulnerable,
keystone, or charismatic species, large numbers (or proportions) of
endemic species, and/or high biodiversity across taxonomic levels (Hogg
et al., 2016). Indeed, the MPA N-BB was created to protect a potentially
sensitive and biodiverse benthic habitat that was only barely known
(Falabella, 2017; Schejter et al., 2016). The benthic community is
featured by high biomass of vulnerable and fragile species (mainly
Porifera, Bryozoa and Cnidaria) that considered with their environment
meet the characteristics of vulnerable marine ecosystems (Schejter \&
Albano, 2021), here defined as sites that present densities of Indicator
Taxa of \textgreater{} 10 kg per 1200 m\textsuperscript{2} (Commission
for the Conservation of Antarctic Marine Living Resources (CCAMLR),
2009). Also, the benthic realm provides habitat to several small-sized
species (López-Gappa et al., 2018; Martin Sirito, 2019; Schejter \&
Bremec, 2019), and has an important role in the life history of fishes
as a food source, refuge and nursery area (Covatti Ale et al., 2022;
Delpiani et al., 2020; Fischer et al., 2022; Florencia et al., 2023;
García Alonso et al., 2018; Matusevich, 2022; Troccoli et al., 2020;
Vazquez et al., 2018). The maintenance of this singular community is
related to local and regional oceanographic processes, including the
circulation of the rich Malvinas (Falkland) current in the area
(Guerrero et al., 1999; Piola \& Gordon, 1989) and the upwelling and
mixing phenomena (Matano et al., 2019). The input of nutrients from the
Malvinas (Falkland) current also supports a diverse plankton community
(Guinder et al., 2020).

Overall, 811 benthic and plankton species have been identified for the
MPA N-BB ecosystem, where 349 were reported for the first time in the
area in recent years (Administración de Parques Nacionales, 2022).
Identifying the main species involved in the maintenance of ecosystem
services and health as well as for management and conservation is
essential. Recently, the structure of the southwestern South Atlantic
Ocean has been proposed to be under a `wasp-waist' control, meaning that
the structure and dynamics of the ecosystem are regulated primarily by
mid-trophic level species (e.g., fishes, crustaceans) (Padovani et al.,
2012; Riccialdelli et al., 2020; Saporiti et al., 2015). In particular,
the ecosystem of the MPA N-BB shows a more pronounced `wasp-waist'
structure, meaning a shorter food chain and a greater trophic overlap
and redundancy, than other sub-Antarctic areas, such as the continental
shelf off Tierra del Fuego. The Fuegian sprat \emph{Sprattus fuegensis}
and longtail southern cod \emph{Patagonotothen ramsayi} are considered
the most plausible `wasp-waist' species (Riccialdelli et al., 2020).

High-latitude marine ecosystems, such as the MPA N-BB, are complex
systems in terms of biodiversity and ecological interactions (Cordone et
al., 2020; Day et al., 2013; Kortsch et al., 2019; Trathan et al.,
2021). Although there is a robust knowledge about the complexity
considering the richness of the benthic and plankton communities in the
MPA N-BB ecosystem (Administración de Parques Nacionales, 2022; Guinder
et al., 2020; Schejter et al., 2016, 2020), a better understanding of
species interactions' complexity and structure is needed. This aspect
can be tackled by analysing one of the most frequent relationships
between species: the predator-prey interaction (Bascompte, 2009). The
sum of predator-prey or trophic interactions of a particular region is
referred to as a food web, representing the roadmap for matter and
energy flow in an ecosystem. In recent years, network approaches have
been successfully applied to study complex high-latitude marine
ecosystems, improving our knowledge on structure, functioning, and
response to environmental/anthropogenic changes (Cordone et al., 2018;
Funes et al., 2022; Kortsch et al., 2015; Marina et al., 2023). Among
anthropogenic threats, it is worth mentioning that contaminants like
mercury and microplastics have been recently reported as important
threats to the MPA N-BB region (Cossi et al., 2021; Di Mauro et al.,
2022; Fioramonti et al., 2022); also fishing vessels are allowed to
operate in the western section of the MPA N-BB (i.e.~Marine National
Reserve category), altering the stocks of commercially important fish
species (Administración de Parques Nacionales, 2022; Martínez et al.,
2021). Moreover, there is a potential hazard related to the effects of
offshore activities (exploration and explotation) to the west of the MPA
N-BB (Administración de Parques Nacionales, 2022).

In the present work, we present the first detailed analysis of the
network of predator-prey interactions, hereafter food web, for the MPA
N-BB ecosystem. For this, we applied a network approach to a highly
resolved food web. The objective was twofold: characterise the food web
in terms of complexity and structure, and identify the species' role in
the network.

\hypertarget{methodology}{%
\section{Methodology}\label{methodology}}

\hypertarget{study-area}{%
\subsection{Study area}\label{study-area}}

The MPAs Namuncurá - Burdwood Bank I and II, created by National Laws
26.875 in 2013 and 27.490 in 2017, comprise a shallow submarine plateau
called Burdwood Bank (BB) and a deep slope that reaches 4000 m in depth,
N-BB I and N-BB II, respectively (Administración de Parques Nacionales,
2022; Tombesi et al., 2020) (Figure 1). They are located 150 km east of
Isla de los Estados and 200 km south of Malvinas/Falkland Islands. The
MPA N-BB I comprises nearly 28,900 km\textsuperscript{2} circumscribed
by the 200 m isobath, between \textasciitilde54º--55ºS and
\textasciitilde56º--62ºW, with a slight slope extended nearly 370 km
east--west. Physical features in the BB are fairly stable, with salinity
averaging 34 all year round and temperature ranging between 4 and 8ºC
(Acha et al., 2004; Guerrero et al., 1999; Piola \& Falabella, 2009).
The BB is surrounded by steep flanks of up to 4000 m depth through which
strong currents circulate (Matano et al., 2019; Piola \& Gordon, 1989;
Reta, 2014). The N-BB II includes such a deep slope, protecting about
32,000 km\textsuperscript{2} (\textasciitilde55º-56ºS,
\textasciitilde58º-62ºW). Intense upwelling and mixing occur in relation
with the slope, entraining deep nutrient-rich waters into the photic
layer (Matano et al., 2019; Piola \& Falabella, 2009) and resulting in a
fairly homogeneous water column both spatially and temporally (Glorioso
\& Flather, 1995; Guerrero et al., 1999; Matano et al., 2019).

Given the evidence collected during several research cruises about the
oceanographic and ecological processes connecting MPAs N-BB I and II
(references in Administración de Parques Nacionales, 2022), a joint
management plan was recently proposed (Administración de Parques
Nacionales, 2022). This is why, the study area of the present work
includes both MPAs.

\begin{figure}
\includegraphics[width=1\linewidth]{MPABurdwood_map} \caption{Marine Protected Areas Namuncurá - Burdwood Bank I (MNR and MNP, northern section) and II (MNR and SMNR, southern section). Acronyms indicate categories according to the management plan: MNR - Marine National Reserve, MNP - Marine National Park and RMNR - Restricted Marine National Reserve.}\label{fig:figure1}
\end{figure}

\hypertarget{network-construction}{%
\subsection{Network construction}\label{network-construction}}

In order to build the network of predator-prey interactions, we reviewed
more than 170 references considering published articles, Ph.D.~theses,
public databases, and reports belonging to 16 research cruises conducted
in the MPAs N-BB I and II during 2014-2019. It is noteworthy that the
sampling effort was greater in the MPA N-BB I. Furthermore, we took into
account personal communications from experts belonging to the working
group of the study area
(https://www.pampazul.gob.ar/tag/banco-burdwood/). The diversity of the
authors' expertise contributing to the present study was a key factor in
enhancing the quality of the network, and inherently improved the
network representation. A list of the references used to build the
network is presented in Supplementary Material (Table S1).

Due to a lack of trophic data resolution for some species inhabiting the
study area, we followed the concept of trophic species, here defined as
follows: taxa collapsed into a single node in the network. In most
cases, we followed this concept when specific data on species, in the
taxonomic sense, were not available. In some cases, we collapsed species
when taxa shared the same set of predators and prey (trophic similarity,
Martinez (1991)), one of the aggregation methods that better preserve
food web functional properties (Gauzens et al., 2013). In addition, for
endemic species (e.g.~bryozoan \emph{Burdwoodipora paguricola}) and
other species with no trophic studies so far, we inferred their feeding
interactions applying a conservative approach that assumes that the set
of prey and predators are at some point preserved in time. In those
cases we gathered information from upper taxonomic levels (i.e.~Genus,
Family, Order, Class, Phylum) as a good proxy variable (Morales-Castilla
et al., 2015; Pomeranz et al., 2019). Details about this can be found in
Supplementary Material (Table S2). Furthermore, we considered non-living
food sources, such as detritus and necromass, as prey species in the
food web context.

With the gathered trophic data, we constructed a matrix of pairwise
interactions; a value of 1 or 0 was assigned to each element a\_ij of
the matrix depending on whether the j-species preyed or not on the
i-species. Then we transformed such a matrix into an oriented graph with
L trophic interactions between S nodes or species. The orientation or
direction of the graph follows the flow of energy and matter in the
network, from prey to predator.

\hypertarget{network-analysis}{%
\subsection{Network analysis}\label{network-analysis}}

We analysed the MPA N-BB network of trophic interactions, or food web,
at two levels: A) network, considering species and interactions of the
whole network; and B) species, considering interactions and species
related to a particular species (Table 1).

The network-level analysis aims to characterise the food web in terms of
complexity and structure. For this, we calculated several network
properties commonly used to describe empirical food webs (Pascual \&
Dunne, 2005): (1) number of species S; (2) number of interactions or
links L; (3) link density L/S; (4) connectance L/S\^{}2; (5) omnivory
Omn; and (6) small-world pattern. In order to explore the small-world
phenomenon, we analysed the characteristic path length (CPL) and the
clustering coefficient (CC). The CPL is the average shortest path length
between all pairs of nodes (Watts \& Strogatz, 1998). Here, CPL was
calculated as the average number of nodes in the shortest path
\(CPL_{Min} (i,j)\) between all pairs of nodes \(S(i,j)\) in a network
averaged over \(S(S-1)/2\) nodes:

\[
CPL = \frac{2}{S(S-1)} \sum_{i = 1}^{S} \sum_{i = 1}^{S} {CPL_{Min}(i,j)}
\] The CC quantifies the local interconnectedness of the network and it
is defined as the fraction of the number of existing links between
neighbours of node \(i\) among all possible links between these
neighbours. In this study, the CC was determined as the average of the
individual clustering coefficients \(CC_i\) of all the nodes in the
network. Individual \(CC_i\) were determined as follows:

\[
CC_i = \frac{2E_i}{K_i(K_i-1)}
\] where \(E_i\) is the effective number of interactions between \(K_i\)
nearest-neighbour nodes of node \(i\) and the maximal possible number of
such interactions (Newman, 2003). To test whether the food web presented
the small-world pattern, we compared the empirical values of CPL and CC
with those resulting from 1000 randomly generated networks with the same
size (S) and number of interactions (L), following the method proposed
by Marina, Saravia, et al. (2018).

Also, we estimated the (7) degree distributions for the food web, prey
and predators, and each functional group (e.g., Amphipoda, Ascidiacea,
Bivalvia, fish, marine mammals, seabirds, among others). The prey and
predator distributions indicate the frequency of prey among predators,
and viceversa; the functional group's degree shows the distribution of
interactions within groups.

The species-level analysis aims to describe the species' role in the
food web. For this, we considered the following properties: betweenness
Btw, closeness Cl, trophic similarity TS, topological role TR, and
trophic level TL (Table 1). Topological roles refer to the fact that
food webs tend to naturally organize in non-random, modular patterns,
where modules are defined as a group of species that interact more
frequently among themselves than with species that are not members of
the module (Guimerà \& Nunes Amaral, 2005). Species can play different
roles in this respect, according to the pattern of interactions within
their own module and/or across modules. We computed the topological role
for each species, classified as module hub, species with a relatively
high number of interactions, but most within its own module; module
specialist, species with relatively few interactions and most within its
own module; module connector, species with relatively few interactions
mainly between modules; and network connector, species with high
connectivity between and within modules (Guimerà \& Nunes Amaral, 2005).

We also studied the relationship between species TL and the other
species properties by performing linear regression analyses. Thus, we
considered the TL as the dependent variable and the given property
(i.e.~betweenness, closeness, trophic similarity) as the independent
variable and obtained the coefficients (slope and intercept) for the
linear model. Models were fitted using the least squares approach. We
also explored the topological role categories with the species TL. These
species-level properties provide an appropriate description of species'
role in empirical complex food webs (Cirtwill et al., 2018).

All network analyses and graphs were performed in R version 4.2.2 (Team,
2022), mainly using `igraph' (Csardi \& Nepusz, 2006) and `multiweb'
(Saravia, 2022) packages. The source code and data are available at
https://github.com/TomasMarina/Banco-Burdwood.

\begin{longtable}[]{@{}
  >{\raggedright\arraybackslash}p{(\columnwidth - 6\tabcolsep) * \real{0.2266}}
  >{\raggedright\arraybackslash}p{(\columnwidth - 6\tabcolsep) * \real{0.2578}}
  >{\raggedright\arraybackslash}p{(\columnwidth - 6\tabcolsep) * \real{0.2578}}
  >{\raggedleft\arraybackslash}p{(\columnwidth - 6\tabcolsep) * \real{0.2578}}@{}}
\caption{List of network and species-level properties analysed,
definitions, and relevant ecological implications related to food web
complexity and structure.}\tabularnewline
\toprule\noalign{}
\begin{minipage}[b]{\linewidth}\raggedright
\textbf{Name}
\end{minipage} & \begin{minipage}[b]{\linewidth}\raggedright
\textbf{Definition}
\end{minipage} & \begin{minipage}[b]{\linewidth}\raggedright
\textbf{Implications}
\end{minipage} & \begin{minipage}[b]{\linewidth}\raggedleft
\textbf{Reference}
\end{minipage} \\
\midrule\noalign{}
\endfirsthead
\toprule\noalign{}
\begin{minipage}[b]{\linewidth}\raggedright
\textbf{Name}
\end{minipage} & \begin{minipage}[b]{\linewidth}\raggedright
\textbf{Definition}
\end{minipage} & \begin{minipage}[b]{\linewidth}\raggedright
\textbf{Implications}
\end{minipage} & \begin{minipage}[b]{\linewidth}\raggedleft
\textbf{Reference}
\end{minipage} \\
\midrule\noalign{}
\endhead
\bottomrule\noalign{}
\endlastfoot
\textbf{Number of species} & Number of trophic species in a food web. &
It represents the species diversity and has implications for the
persistence of the ecosystem. & May 1973, Tilman 1996 \\
\textbf{Number of interactions} & Total number of trophic interactions
in a food web. & It represents the number of pathways along which matter
and energy can flow. & Dunne et al.~2002 \\
\textbf{Link density} & Ratio of interactions to species in a food web &
It represents the average number of interactions per species; informs
about how connected species are in the food-web. & Dunne et al.~2002 \\
\textbf{Connectance} & Proportion of potential links among species that
are actually realized. Range = 0 - 1. & It measures the probability of
interactions and is a fundamental measure of network complexity.
Connectance can be negatively or positively associated with food web
robustness, depending on the network structure (random vs non-random) or
how the strength of the interactions are distributed. & Martinez 1992 \\
\textbf{Degree distribution} & Frequency of trophic species that have k
or more interactions. & It suggests on the vulnerability of complex food
webs against random failures and intentional attacks (i.e. species
extinctions). & Albert \& Barabási 2002 \\
\textbf{Omnivory} & Species feeding on prey from more than one trophic
level. & It influences food web's stability; intermediate levels of
omnivory may stabilize it and may diffuse top-down effects thus reduce
the probability of trophic cascades. & McCann \& Hastings 1997 \\
\textbf{Small-world pattern} & A network with short path length
(distance between nodes) and high clustering coefficient (formation of
compartments) compared to random networks. & Consequences of this
structural pattern in food webs are of great importance in recognizing
evolutionary paths and the vulnerability to perturbations. & Watts \&
Strogatz 1998, Montoya \& Solé 2002 \\
\textbf{Betweenness} & Number of shortest paths going through a species.
& Species with high betweenness act as ``bridges''; if removed, would
have rapidly spreading effects in the food web. & Freeman 1978, Lai et
al.~2012 \\
\textbf{Closeness} & Number of steps required to reach every other
species from a given species. & The removal of a species with high
closeness will affect the most other species in the food web. & Freeman
1978, Lai et al.~2012 \\
\textbf{Trophic similarity} & Trophic overlap based on shared and unique
resources (prey) and consumers (predators). & It measures one of the
most important aspects of species' niches, the trophic niche, and
functional aspects of biodiversity. & Martinez 1992 \\
\textbf{Topological role} & Species role according to interactions
within and across modules (subgroups of species). & Four roles are
defined: module hub, module specialist, module connector and network
connector. Network connector and module connector roles maintain the
connectivity of the food web. & Guimera \& Nunes Amaral 2005 \\
\end{longtable}

\hypertarget{results}{%
\section{Results}\label{results}}

\hypertarget{network-level-properties}{%
\subsection{Network-level properties}\label{network-level-properties}}

In terms of complexity, the MPA Namuncurá - Burdwood Bank food web
consisted of 1788 predator-prey interactions and 379 species, where 93\%
of them were defined at the species taxonomical level (Figure 2, Table
S2). The food web presented a link density (e.g., the average number of
interactions per species) of 4.72, and a connectance of 0.01. Almost
half of the consumers were omnivores (0.48), feeding on sources at
different trophic levels. The food web displayed a small-world pattern,
meaning that the path length was lower and the clustering coefficient
higher than the random networks (Table 2).

\newpage

\begin{figure}

{\centering \includegraphics{MS_Burdwood_foodweb_files/figure-latex/figure2-1} 

}

\caption{Graph of the food web for the MPA Namuncurá - Burdwood Bank. Circles represent species and arrows trophic interactions. Circle diameter is relative to the number of interactions. Colour gradient indicates the trophic level.}\label{fig:figure2}
\end{figure}

\begin{table}

\caption{\label{tab:table2}Network-level properties of the MPA Namuncurá - Burdwood Bank food web. CPL: Characteristic Path Length; CC: Clustering Coefficient; SW: Small-World pattern. See table 1 for definitions and ecological relevance.}
\centering
\begin{tabular}[t]{r|r|r|r|r|r|r|l}
\hline
\textbf{Species} & \textbf{Interactions} & \textbf{Density} & \textbf{Connectance} & \textbf{Omnivory} & \textbf{CPL} & \textbf{CC} & \textbf{SW}\\
\hline
379 & 1788 & 4.72 & 0.01 & 0.49 & 2.99 & 0.08 & True\\
\hline
\end{tabular}
\end{table}

The degree distribution of the food web showed an asymmetric frequency
in the number of interactions, where most of the species had a
relatively low number of interactions and few species concentrated most
of them (Figure 3A). The distribution of prey among predators showed
that most consumers fed on a low number of prey whereas few had multiple
prey (Figure 3B). The top-five predators in number of prey were:
yellowfin notothen \emph{Patagonotothen guntheri} (Notothenioid fish, 50
prey), rock cod \emph{Patagonotothen ramsayi} (Notothenioid fish, 49
prey), broad nose skate \emph{Bathyraja brachyurops} (Chondrichthyan, 33
prey), Patagonian toothfish \emph{Dissostichus eleginoides}
(Notothenioid fish, 30 prey), and graytail skate \emph{Bathyraja
griseocauda} (Chondrichthyan, 28 prey). Following the same distribution
pattern, few prey presented multiple predators (Figure 3C). The top-five
prey (or food sources) in number of predators were: Detritus
(Non-living, 153 predators), the three categories of Diatoms considered
(benthic, centric and pennate, 72.5 predators on average), and species
of the genus \emph{Euphausia} (Zooplankton, 46 predators). Finally,
taking into account the interactions within each functional group, most
interactions were concentrated in a few species (Figure 3D). The most
evident species were: \emph{Doryteuthis gahi} (Cephalopoda),
\emph{Grimothea (=Munida) gregaria} (Decapoda), \emph{Patagonotothen
ramsayi}, \emph{Patagonotothen guntheri} and \emph{Dissostichus
eleginoides} (bentho-pelagic fish), \emph{Sprattus fuegensis} and
\emph{Micromesistius australis} (pelagic fish), and species of
\emph{Euphausia} and \emph{Themisto gaudichaudii} (Zooplankton).
Overall, there is an evident asymmetry in the distribution of
interactions among species at different levels in the MPA N-BB food web.

A list of the distribution of interactions per species is presented in
Supplementary Material (Table S3).

\begin{figure}

{\centering \includegraphics{MS_Burdwood_foodweb_files/figure-latex/figure3-1} 

}

\caption{Degree distributions for the (A) food web, for (B) prey among predators, (C) predators among prey, and (D) for each functional group. Groups are vertically ordered by increasing trophic level (following coloration of figure 2); groups with less than 3 species were not plotted (e.g., pelagic fish). All functional groups and the species that comprise them are shown in Supplementary Material (Table S3).}\label{fig:figure3}
\end{figure}

\hypertarget{species-level-properties}{%
\subsection{Species-level properties}\label{species-level-properties}}

We found different relationships between the species trophic level (TL)
and the rest of the analysed species-level properties (Figure 4A-D). The
most evident significant relationship was with trophic similarity,
i.e.~the higher the species' TL, the lower the trophic similarity or the
higher the uniqueness in terms of trophic role (Figure 4C). Here it is
noteworthy to highlight those high-trophic level species (TL
\textgreater{} 3.1) with low values of trophic similarity:
\emph{Bathyraja macloviana} and \emph{Squalus acanthias}
(Chondrichthyans), \emph{Diplopteraster clarki} and \emph{Pteraster} sp.
(echinoderms), \emph{Daption capense} and \emph{Eudyptes chrysocome}
(seabirds), Ziphiidae and \emph{Lagenorhynchus cruciger} (marine
mammals) (Table S3).

We also found a significant negative relationship between TL and
closeness, however less evident, meaning that low-TL species are
relatively closer to any other species in the food web (Figure 4B).
Detritus, species of genera \emph{Calanus} and \emph{Euphausia}, and
Foraminifera, all with TL \textless{} 3, registered the highest
closeness values (Table S3).

Notably, species of mid-TLs (3-4.2) showed the highest values of
betweenness, meaning that those species participated in the highest
number of shortest paths between species (Figure 4A). The following are
the species with the highest values (descending order):
\emph{Patagonotothen ramsayi}, \emph{Salilota australis},
\emph{Dissostichus eleginoides} (fishes), \emph{Doryteuthis gahi}
(Cephalopoda), and \emph{Patagonotothen guntheri} (Notothenioid fish)
(Table S3).

Considering the topological role, `module specialist' species were the
most frequent and presented a wide TL range (1 - 4.78), as well as
`module hub' species (TL = 1 - 3.92); `module connector' was constrained
to mid-TLs (2 - 3.86); and `network connector', was represented by only
one trophic species: detritus (Figure 4D, see Figure S2 for species'
topological roles in a food web graph framework). Here it is important
to highlight the two latter topological roles because they are
responsible for linking modules and maintaining the connectivity of the
food web: 42 species (1 network connector + 41 module connectors) from
19 different functional groups with a TL range = 1 - 3.86. The 41
species with a module connector role represented these functional
groups: Amphipoda, Bivalvia, Brachiopoda, Bryozoa, Hydrozoa (as
`Cnidaria\_benthic'), Copepoda, Cumacea, Decapoda, Echinodermata, fish
(bentho-pelagic and demersal Osteichthyes, and Chondrychthyes),
Foraminifera, Polychaeta, Porifera, Pycnogonida (as `Benthos\_Misc') and
zooplankton (see Supplementary Material Table S3 for the identity of the
species).

An exhaustive list of the species-level properties is presented in
Supplementary Material (Table S3).

\begin{figure}

{\centering \includegraphics{MS_Burdwood_foodweb_files/figure-latex/figure4-1} 

}

\caption{Species-level properties by trophic level: (A) betweenness, (B) closeness, (C) trophic similarity, and (D) topological role. Each point represents a species. Linear regressions for betweenness ($y = 74.97x - 117.35, R^2 = 0.05, p-value < 0.01$), closeness ($y = 9.33e-06x - 9.31e-4, R^2 = 0.003, p-value = 0.15$) and trophic similarity ($y = -0.02x + 0.11 , R^2 = 0.07, p-value < 0.01$). Note that for panels A, B and C only species with TLs equal or greater than 2 were considered.}\label{fig:figure4}
\end{figure}

\hypertarget{discussion}{%
\section{Discussion}\label{discussion}}

\hypertarget{the-food-web-of-the-mpa-namuncuruxe1---burdwood-bank-ecosystem}{%
\subsection{The food web of the MPA Namuncurá - Burdwood Bank
ecosystem}\label{the-food-web-of-the-mpa-namuncuruxe1---burdwood-bank-ecosystem}}

The food web of the MPA N-BB ecosystem analysed in this study is one of
the most highly-resolved networks of trophic interactions ever studied,
not only for a high-latitude open-ocean ecosystem but also for any
marine protected area worldwide to our knowledge. It is of paramount
importance to consider the complexity of species interactions in order
to gain insights into the structure and functioning of the ecosystem,
since the aggregation of species might mask food web properties and
produce type II errors (false positives) (Gauzens et al., 2013;
Martinez, 1993).

Food web connectance is a feature that resumes the complexity of the
network, but more importantly, it is an emergent property of pairwise
species interactions (Poisot \& Gravel, 2014). It contains information
regarding how interactions within an ecological network are distributed
and predicts reasonably well key dynamical properties of ecological
networks (Dunne et al., 2002c). Complex marine food webs (i.e.~with more
than 25 trophic species) show connectance values ranging from 0.01 -
0.27 (Marina, Saravia, et al., 2018). In particular, food webs from
high-latitude regions tend to exhibit a connectance closer to the
minimum (between 0.01 and 0.05) (Kortsch et al., 2015; Rodriguez et al.,
2022; Santana et al., 2013). Whether food webs display a low or a high
connectance helps to better comprehend ecosystem's synthetic properties
like robustness. In this sense, empirical analyses support the notion
that highly-connected ecological networks are robust against external
perturbations such as the introduction of new (e.g., invasive) species
(Smith-Ramesh et al., 2017) as well as species removal (e.g., local
extinction) (Dunne et al., 2002a; Montoya \& Solé, 2003). The
connectance of the food web of the MPA Namuncurá - Burdwood Bank (0.01)
is one of the lowest reported so far for these regions; in particular,
it appears to be much lower than that of Beagle Channel (0.05), an
adjacent coastal area (Rodriguez et al., 2022).

The degree distribution, the distribution of the number of interactions
per species, is the core of the structure of species interactions, which
influences the opportunities for multiple species to persist in the long
term and, therefore, their coexistence (Godoy et al., 2018). The food
web for the MPA N-BB presents an asymmetric degree distribution. This
pattern was identified at different levels of analysis: food web,
predator, prey, and functional group. Such asymmetry is a well-known
feature in empirical complex food webs in particular (Dunne et al.,
2002c; Montoya \& Solé, 2003; Stouffer et al., 2005), and has received
great attention in complex networks in general (Albert \& Barabási,
2002; Newman, 2003). The degree distribution affects the resilience of
complex food webs against random failures and pressure on a particular
component of the web: food webs showing right-skewed distributions, like
the one described in this study, are more vulnerable to the removal of
the most connected species or hubs, with the potential of producing
secondary extinctions and a catastrophic fragmentation of the network
(Albert et al., 2000; Dunne et al., 2002a; Eklöf \& Ebenman, 2006).

It is suggested that the small-world pattern, i.e., a network with short
path length and high clustering coefficient, is not frequent in complex
marine food webs, mainly due to a low clustering coefficient compared to
random networks (Dunne et al., 2002b; Marina, Saravia, et al., 2018).
However, the food web of the MPA N-BB does display a small-world
pattern. Consequences of this could be of great importance in
recognizing species evolutionary paths and the vulnerability to
perturbations (Montoya \& Solé, 2002). On the one hand, a short path
length implies a rapid spread of an impact (e.g., contaminant,
population fluctuation, local extinction) throughout the network but, at
the same time, more potentially adaptive dynamics in the face of
external perturbations (Montoya \& Solé, 2002; Williams et al., 2002).
On the other hand, a high clustering coefficient indicates the formation
of subnetworks composed only by the neighbours of particular species.
This translates into a greater resistance of the network due to the
confinement of perturbations mainly within subnetworks and not spreading
between them (Heer et al., 2020; Kortsch et al., 2019). Overall, a
small-world topology provides ecological networks with greater
resilience and resistance (Bornatowski et al., 2017; Dormann et al.,
2017).

Omnivory acts as a buffer to changes as the ecosystem presents
alternative energy pathways in the face of perturbations, i.e., reducing
the risk of cascading extinctions following the primary loss of species
(Borrvall et al., 2000). Omnivores are species able to adapt faster and
to a broader range of environmental conditions by changing their
foraging habits to feed on the most abundant prey (Fagan, 1997).
Furthermore, omnivory can be analysed from the interaction point of
view: theoretical studies have identified omnivorous interactions as a
possible candidate for a keystone interaction, sensu Kadoya et al.
(2018), highlighting the importance of omnivory in stabilizing food web
dynamics (McCann \& Hastings, 1997; Neutel et al., 2002). The high
proportion of omnivory in the food web of the MPA N-BB suggests that the
network might be robust to variations in prey abundances, which could
increase food web's persistence and stability (Stouffer \& Bascompte,
2010).

In summary, the food web of the MPA N-BB presents a combination of
network properties that makes it unique in terms of network resolution,
complexity, and structural pattern. All this suggests that the food web
might be fragile to external perturbations targeting highly connected
species, which in turn coincides to be commercial exploited species as
fishes (Laptikhovsky et al., 2013; Martínez et al., 2015; Winter \&
Arkhipkin, 2023). However, structural properties might provide
resilience and resistance with the final outcome of a rearranged
structure maintaining its functions.

\hypertarget{dominant-consumers-and-food-sources}{%
\subsection{Dominant consumers and food
sources}\label{dominant-consumers-and-food-sources}}

The degree distribution allows identifying important species, such as
potential keystone species (i.e.~highly connected) (Dunne et al., 2002a;
Solé \& Montoya, 2001), generalist/specialist species, and dominant food
sources (Kondoh et al., 2010).

We have identified that most of the consumers in the food web of the MPA
N-BB either have a narrow diet or are specialists, while few present a
broad or generalist diet. The most evident generalist species are
\emph{Patagonotothen guntheri} (Covatti Ale et al., 2022), \emph{P.
ramsayi} (Fischer et al., 2022), juveniles of \emph{Dissostichus
eleginoides} (Troccoli et al., 2020), \emph{Bathyraja brachyurops}
(Belleggia et al., 2008), and \emph{B. griseocauda} (Bellegia et al.,
2014), with more than 25 potential prey. Since these species present
mid-trophic positions in the food web (with the exception of adults of
\emph{Dissostichus eleginoides} that are top predators), acting as
predator and prey, they might be important links between lower and
higher trophic levels. This result is in agreement with the sole
analysis, using stable isotopes, that exists so far for the trophic
structure of the MPA N-BB (Riccialdelli et al., 2020), and resembles
other high-latitude marine systems of the Southwest Atlantic and
Antarctic regions (Arkhipkin \& Laptikhovsky, 2013; Marina, Salinas, et
al., 2018). The importance of these particular generalist species also
arises since they feed in the benthic and pelagic habitats (Covatti Ale
et al., 2022; Fischer et al., 2022; Troccoli et al., 2020), linking
these realms and contributing to the vertical carbon flow.

On the other hand, a low number of prey are consumed by many predators
in the food web of the MPA N-BB. This suggests that there are dominant
food sources on which most consumers depend and from where the ecosystem
energy is being transferred to the upper trophic levels. The most
demanded source we identified in this study (i.e.~detritus) supports the
abundant benthic community of filter-feeders (Schejter et al., 2016),
components of the animal forest (Schejter et al., 2020), likely feeding
on detritus that is constantly resuspended from the bottom (Martin \&
Flores Melo, 2021). Furthermore, we found that the second and third-most
consumed prey were diatoms and species of \emph{Euphausia},
respectively, which are essential sources for the diverse zooplankton
community (Spinelli et al., 2020), mid-TL consumers like the Fuegian
sprat \emph{Sprattus fuegensis} (Padovani et al., 2021) and
\emph{Patagonotothen ramsayi} (Fischer et al., 2022), and top predators
such as the black-browed and grey-headed albatrosses (\emph{Thalassarche
melanophris} and \emph{Thalassarche chrysostoma}, respectively) (Catry
et al., 2004), and baleen whales (species of the genera
\emph{Balaenoptera} and \emph{Eubalaena}) (Valenzuela et al., 2018).

\hypertarget{species-role-related-to-their-trophic-level}{%
\subsection{Species' role related to their trophic
level}\label{species-role-related-to-their-trophic-level}}

Describing species' roles in food webs provides a toolbox to assess the
significance of species in terms of community's functioning and overall
stability (Cirtwill et al., 2018; Thébault \& Fontaine, 2010). We used a
range of descriptors to characterise the dynamic and multifaceted nature
of the species forming the MPA N-BB food web.

Closeness and betweenness are defined as ``mesoscale'' properties
because they consider direct and indirect interactions, therefore
describing the focal species' ability to influence the rest of the
species of the food web (Lai et al., 2012). Closeness quantifies how
many steps away species \(i\) is from all other species in the food web,
and is proportional to how rapidly the indirect effects of the focal
species can spread to other species in the network (Scotti \& Jordán,
2010). In the food web of the MPA N-BB, low-TL consumers arise as
important in this regard: species of the zooplankton community,
\emph{Calanus} and \emph{Euphausia}, \emph{Zygochlamys patagonica}
(Bivalvia), and Brachiopoda. Any perturbation affecting these species,
such as the recently confirmed contaminants mercury (Fioramonti et al.,
2022) and microplastics (Cossi et al., 2021; Di Mauro et al., 2022),
should be of concern since it might reach many other species in the food
web. Otherwise, betweenness measures the number of shortest paths
between species, providing information on the importance of species as
``bridges'' for energy transfer: a species with high betweenness takes
part in more food chains and therefore affects more energy flows (Scotti
\& Jordán, 2010). We have identified the longtail southern cod
\emph{Patagonotothen ramsayi} as the most important species in this
sense. Moreover, in light of our analysis, other species like the
Patagonian toothfish \emph{Dissostichus eleginoides} (juveniles), the
Patagonian cod \emph{Salilota australis}, the yellowfin notothenioid
\emph{Patagonotothen guntheri}, and the Patagonian longfin squid
\emph{Doryteuthis gahi} should be considered as relevant in the energy
transfer in the ecosystem. All these species have a mid-trophic position
in the food web, supporting the `wasp-waist' control hypothesis for the
MPA N-BB (Riccialdelli et al., 2020).

Ecosystems with a pronounced `wasp-waist' structure are suggested to
present a high trophic redundancy, since many species would show similar
trophic habits (Cury et al., 2000). The significant negative
relationship between trophic similarity and trophic level enhances the
hypothesis of functional similarity at low and mid-TL species compared
to higher TL species for the MPA N-BB food web (Riccialdelli et al.,
2020). At the same time, our results highlight the uniqueness in terms
of the trophic role of high-TL predators. Here, not only the expected
pelagic animals such as marine mammals and seabirds arise as relevant,
but also demersal vertebrate (chondrichthyans \emph{Bathyraja
macloviana} and \emph{Squalus acanthias}) and benthic invertebrate
species (echinoderms \emph{Diplopteraster clarki} and \emph{Pteraster}
sp.) are noteworthy. The role that such species play in the MPA N-BB
ecosystem is unique and perturbations on them might result in
unprecedented changes at the trophic structure and functioning level. In
this regard, we should mention the potential threat of the fisheries
operating in the western section of the MPA N-BB, where this activity is
allowed and mostly focuses on the Patagonian toothfish
\emph{Dissostichus eleginoides} and the southern blue whiting
\emph{Micromesistius australis} (Martínez et al., 2015). Although the
fishing effort is concentrated outside the limits of the MPA N-BB, the
impact on the MPA ecosystem should not be neglected (Martínez et al.,
2021).

Species' role can also be assessed in a module-based context. Among the
varying numbers of topological roles in which species can be divided,
two are remarkable: `module connector' and `network connector'. Here,
our results point out that there are several species, belonging to a
wide range of trophic positions (1 to 3.86) and representing 17
different functional groups, that should be considered as influential
species for the connectivity of the food web. Thus, we propose that the
diversity of species (benthic and pelagic) maintains the connectivity of
the food web, therefore contributing to the trophic structure and
ecosystem's stability.

\hypertarget{caveats-and-future-perspectives}{%
\subsection{Caveats and future
perspectives}\label{caveats-and-future-perspectives}}

The food web studied in the present work might be more representative of
the shallow ecosystem of the submarine plateau called Burdwood Bank, on
which most of the research was focused as the MPA N-BB I was first
created. This is related to the sampling effort that was conducted
during the research cruises in the former MPA compared to the MPA N-BB
II (i.e.~deep flanks to the south). As a consequence, most of the data
we used to build the network come from studies performed in the MPA N-BB
I. Despite this fact, we decided to build the food web considering both
MPAs due to the tight oceanographic and ecological connection that
exists among them (Administración de Parques Nacionales, 2022 and
references therein).

It's important to mark that we did not consider quantitative data
(i.e.~abundance, biomass) to assess the species' role in the food web.
Although there exists such data for some species (Schejter \& Albano,
2021), it would not be possible to include it in the food web framework
described here due to a taxonomical resolution mismatch. In this regard,
we should mention the case of \emph{Zygochlamys patagonica} (Bivalvia)
and Brachiopoda that are highlighted by our species-level analyses
though they have been found in low abundances in the area (Schejter \&
Albano, 2021).

Some species of sessile suspension feeders in high-latitude marine
ecosystems, such as sponges, ascidians and octocorals, avoid predation
by producing secondary metabolites that function as a chemical defense
(Moles et al., 2015; Núñez-Pons et al., 2010; Prieto et al., 2022).
Although this was not yet recorded at the MPA N-BB, there are a few
studies that reported it in other locations in species that inhabit the
MPA N-BB (Rojo de Almeida et al., 2010).

The MPA N-BB I presents complex oceanographic conditions that generate
an internal spatial heterogeneity, mainly along its longitudinal axis
(Matano et al., 2019). So far this heterogeneity has been reflected in
phytoplankton and zooplankton communities (Bértola et al., 2018; García
Alonso et al., 2020; Spinelli et al., 2020), and in fish assemblages
(Delpiani et al., 2020). Moreover, seasonal variations also occur in
some physical and biological aspects of the MPA N-BB I (García Alonso et
al., 2018; Matano et al., 2019). Considering both MPAs (N-BB I and II),
a seasonal variation in the community composition of marine mammals and
seabirds was recorded recently (Dellabianca et al., 2023). The spatial
and seasonal variations in the plankton community might affect the
energy and matter flow to higher levels of the food web. This has been
recently studied in the vicinity of the MPA N-BB I, in the Beagle
Channel, where a differential energy flow pattern of the plankton
community has been recognised in two micro-basins of the Channel
separated by a sill, each with different physicochemical properties
(Giesecke et al., 2021), nutrient concentration (Latorre et al., 2023)
as well as in the dominant component of the plankton community (Bruno et
al., 2023; Presta et al., 2023). Although we were aware of the above, we
decided to characterise a food web representing the whole MPA N-BB I
year round since this is the first study of its type in the area.

Taking into account the mentioned caveats, and with the aim of improving
the knowledge regarding the structure, functioning and stability of the
MPA N-BB, we suggest that the future perspectives should: 1) incorporate
spatial heterogeneity among MPA N-BB I and II (Schejter \& Albano,
2021), which might lead to distinct food web properties in terms of
structure and functioning (Cordone et al., 2020; Kortsch et al., 2019);
2) include species traits, like body size and mass, since they are known
to be important drivers in predator-prey interactions (Brose et al.,
2019); 3) simulate the anthropogenic impacts already present in the MPA
N-BB ecosystem (e.g.~microplastics, mercury) (Cossi et al., 2021; Di
Mauro et al., 2022; Fioramonti et al., 2022) as perturbations within the
framework of the described complex food web; and 4) estimate the
interaction strength of each predator-prey relationship in the food web
considering species and interaction traits (i.e.~body size, body mass,
interaction dimensionality), and species density data (Nilsson \&
McCann, 2016; Pawar et al., 2012).

\hypertarget{conclusion}{%
\section{Conclusion}\label{conclusion}}

We compiled information on the species and trophic diversity of the
oceanic Marine Protected Area Namuncurá - Burdwood Bank, generating an
unprecedented, well-resolved network of trophic interactions for a
sub-Antarctic ecosystem, identifying the complexity and structure of the
system, and the main species role in a network framework. Particular
properties at the network level allowed us to identify the ecosystem's
vulnerability and potential response to perturbations in the presence of
highly-connected species, with a rearranged structure maintaining their
functions due to its potential resilience and resistance.

We identified several species as important regarding different aspects
of trophic structure and functioning, and response to perturbations
(i.e.~environmental/anthropogenic changes). On the one hand, we suggest
that generalist species, mainly fishes, play a crucial role in the
ecosystem's bentho-pelagic coupling process. At the same time, we
propose that other species besides the longtail southern cod
\emph{Patagonotothen ramsayi} and the Fueguian sprat \emph{Sprattus
fuegensis} should be considered relevant energy transfers for the
ecosystem. Finally, we argue that it is the diversity of species,
representing the benthic and pelagic habitats, that maintains the
connectivity of the food web against perturbations, therefore
contributing to the structure and stability of the ecosystem.

\hypertarget{acknowledgements}{%
\section{Acknowledgements}\label{acknowledgements}}

We are indebted to all those experts of the working group `Banco
Burdwood' who humbly provided their knowledge to enhance the quality of
the present research. Although most of them are authors of the present
work, it is worth to mention the following researchers: Brenda L. Doti
(IBBEA, CONICET-UBA; Universidad de Buenos Aires, Argentina), Sofía L.
Callá (Museo Argentino de Ciencias Naturales ``Bernardino Rivadavia'',
Argentina), Sandra Gordillo (IDACOR-CONICET; Universidad Nacional de
Córdoba, Argentina), Mariano I. Martinez (Museo Argentino de Ciencias
Naturales ``Bernardino Rivadavia'', Argentina) and Luciano Padovani
(Instituto Nacional de Investigación y Desarrollo Pesquero, INIDEP,
Argentina). We thank the MPA Namuncurá − Burdwood Bank administration.
Research cruises were funded by national funds by the Law 26.875. This
study was funded by Consejo Nacional de Investigaciones Científicas y
Técnicas (CONICET) and Agencia Nacional de Promoción Científica y
Tecnológica (PICT 2020-SERIEA-01617), Argentina. This work is
contribution no. XX of the MPA Namuncurá (Law 26.875).

\hypertarget{references}{%
\section*{References}\label{references}}
\addcontentsline{toc}{section}{References}

\hypertarget{refs}{}
\begin{CSLReferences}{1}{0}
\leavevmode\vadjust pre{\hypertarget{ref-Acha2004}{}}%
Acha, E. M., Mianzan, H. W., Guerrero, R. A., Favero, M., \& Bava, J.
(2004). Marine fronts at the continental shelves of austral {South
America}: {Physical} and ecological processes. \emph{Journal of Marine
Systems}, \emph{44}(1), 83--105.
\url{https://doi.org/10.1016/j.jmarsys.2003.09.005}

\leavevmode\vadjust pre{\hypertarget{ref-AdministraciondeParquesNacionales2022}{}}%
Administración de Parques Nacionales. (2022). \emph{{Plan de gestión AMP
Namuncurá Banco Burdwood}}. {Dirección Nacional de Áreas Marinas
Protegidas (DNAMP), Argentina}.

\leavevmode\vadjust pre{\hypertarget{ref-Albert2002}{}}%
Albert, R., \& Barabási, A.-L. (2002). Statistical mechanics of complex
networks. \emph{Reviews of Modern Physics}, \emph{74}(1), 47--97.
\url{https://doi.org/10.1103/RevModPhys.74.47}

\leavevmode\vadjust pre{\hypertarget{ref-Albert2000}{}}%
Albert, R., Jeong, H., \& Barabási, A.-L. (2000). Error and attack
tolerance of complex networks. \emph{Nature}, \emph{406}(6794),
378--382. \url{https://doi.org/10.1038/35019019}

\leavevmode\vadjust pre{\hypertarget{ref-Arkhipkin2013}{}}%
Arkhipkin, A., \& Laptikhovsky, V. (2013). From gelatinous to muscle
food chain: Rock cod {Patagonotothen} ramsayi recycles coelenterate and
tunicate resources on the {Patagonian Shelf}. \emph{Journal of Fish
Biology}, \emph{83}(5), 1210--1220.
\url{https://doi.org/10.1111/jfb.12217}

\leavevmode\vadjust pre{\hypertarget{ref-Bascompte2009}{}}%
Bascompte, J. (2009). Disentangling the {Web} of {Life}. \emph{Science},
\emph{325}(5939), 416--419.
\url{https://doi.org/10.1126/science.1170749}

\leavevmode\vadjust pre{\hypertarget{ref-Belleggia2008a}{}}%
Belleggia, M., Mabragaña, E., Figueroa, D. E., Scenna, L. B., Barbini,
S. A., \& Astarloa, J. M. D. de. (2008). Food habits of the broad nose
skate, {\emph{Bathyraja}}{ \emph{Brachyurops}} ({Chondrichthyes},
{Rajidae}), in the south-west {Atlantic}. \emph{Scientia Marina},
\emph{72}(4), 701--710.
\url{https://doi.org/10.3989/scimar.2008.72n4701}

\leavevmode\vadjust pre{\hypertarget{ref-Bellegia2014}{}}%
Bellegia, M., Scenna, L., Barbini, S. A., Figueroa, D. E., \& Díaz de
Astarloa, J. M. (2014). \emph{The diets of four {Bathyraja} skates
({Elasmobranchii}, {Arhynchobatidae}) from the {Southwest Atlantic}}.
\url{https://doi.org/10.26028/CYBIUM/2014-384-012}

\leavevmode\vadjust pre{\hypertarget{ref-Bertola2018}{}}%
Bértola, G., Olguín Salinas, H., \& Alder, V. A. (2018). Distribución
espacial de {Rhizosolenia} crassa, \textquestiondown especie clave del
banco burdwood? \emph{Libro de Resúmenes {X Jornadas Nacionales} de
{Ciencias} Del {Mar}}.

\leavevmode\vadjust pre{\hypertarget{ref-Bornatowski2017}{}}%
Bornatowski, H., Barreto, R., Navia, A. F., \& de Amorim, A. F. (2017).
Topological redundancy and {``small-world''} patterns in a food web in a
subtropical ecosystem of {Brazil}. \emph{Marine Ecology}, \emph{38}(2),
e12407. \url{https://doi.org/10.1111/maec.12407}

\leavevmode\vadjust pre{\hypertarget{ref-Borrvall2000}{}}%
Borrvall, C., Ebenman, B., \& Tomas Jonsson, T. J. (2000). Biodiversity
lessens the risk of cascading extinction in model food webs.
\emph{Ecology Letters}, \emph{3}(2), 131--136.
\url{https://doi.org/10.1046/j.1461-0248.2000.00130.x}

\leavevmode\vadjust pre{\hypertarget{ref-Brose2019}{}}%
Brose, U., Archambault, P., Barnes, A. D., Bersier, L.-F., Boy, T.,
Canning-Clode, J., Conti, E., Dias, M., Digel, C., Dissanayake, A.,
Flores, A. A. V., Fussmann, K., Gauzens, B., Gray, C., Häussler, J.,
Hirt, M. R., Jacob, U., Jochum, M., Kéfi, S., \ldots{} Iles, A. C.
(2019). Predator traits determine food-web architecture across
ecosystems. \emph{Nature Ecology \& Evolution}, \emph{3}(6), 919--927.
\url{https://doi.org/10.1038/s41559-019-0899-x}

\leavevmode\vadjust pre{\hypertarget{ref-Bruno2023a}{}}%
Bruno, D. O., Valencia-Carrasco, C., Paci, M. A., Leonarduzzi, E.,
Castro, L., Riccialdelli, L., Iachetti, C. M., Cadaillon, A., Giesecke,
R., Schloss, I. R., Berghoff, C. F., Martín, J., Diez, M., Cabreira, A.,
Presta, M. L., Capitanio, F. L., \& Boy, C. C. (2023). Spring plankton
energy content by size classes in two contrasting environments of a high
latitude ecosystem: {The Beagle Channel}. \emph{Journal of Marine
Systems}, \emph{240}, 103876.
\url{https://doi.org/10.1016/j.jmarsys.2023.103876}

\leavevmode\vadjust pre{\hypertarget{ref-Catry2004}{}}%
Catry, P., Phillips, R. A., Phalan, B., Silk, J. R. D., \& Croxall, J.
P. (2004). Foraging strategies of grey-headed albatrosses {Thalassarche}
chrysostoma: Integration of movements, activity and feeding events.
\emph{Marine Ecology Progress Series}, \emph{280}, 261--273.
\url{https://doi.org/10.3354/meps280261}

\leavevmode\vadjust pre{\hypertarget{ref-Cirtwill2018}{}}%
Cirtwill, A. R., Dalla Riva, G. V., Gaiarsa, M. P., Bimler, M. D.,
Cagua, E. F., Coux, C., \& Dehling, D. M. (2018). A review of species
role concepts in food webs. \emph{Food Webs}, \emph{16}, e00093.
\url{https://doi.org/10.1016/j.fooweb.2018.e00093}

\leavevmode\vadjust pre{\hypertarget{ref-CommissionfortheConservationofAntarcticMarineLivingResourcesCCAMLR2009}{}}%
Commission for the Conservation of Antarctic Marine Living Resources
(CCAMLR). (2009). \emph{Vulnerable {Marine Ecosystem} taxa
identification guide}. {Commission for the Conservation of Antarctic
Marine Living Resources}.

\leavevmode\vadjust pre{\hypertarget{ref-Cordone2018}{}}%
Cordone, G., Marina, T. I., Salinas, V., Doyle, S. R., Saravia, L. A.,
\& Momo, F. R. (2018). Effects of macroalgae loss in an {Antarctic}
marine food web: Applying extinction thresholds to food web studies.
\emph{PeerJ}, \emph{6}, e5531. \url{https://doi.org/10.7717/peerj.5531}

\leavevmode\vadjust pre{\hypertarget{ref-Cordone2020}{}}%
Cordone, G., Salinas, V., Marina, T. I., Doyle, S. R., Pasotti, F.,
Saravia, L. A., \& Momo, F. R. (2020). Green vs brown food web:
{Effects} of habitat type on multidimensional stability proxies for a
highly-resolved {Antarctic} food web. \emph{Food Webs}, \emph{25},
e00166. \url{https://doi.org/10.1016/j.fooweb.2020.e00166}

\leavevmode\vadjust pre{\hypertarget{ref-Cossi2021}{}}%
Cossi, P. F., Ojeda, M., Chiesa, I. L., Rimondino, G. N., Fraysse, C.,
Calcagno, J., \& Pérez, A. F. (2021). First evidence of microplastics in
the {Marine Protected Area Namuncurá} at {Burdwood Bank}, {Argentina}: A
study on {Henricia} obesa and {Odontaster} penicillatus
({Echinodermata}: {Asteroidea}). \emph{Polar Biology}, \emph{44}(12),
2277--2287. \url{https://doi.org/10.1007/s00300-021-02959-5}

\leavevmode\vadjust pre{\hypertarget{ref-CovattiAle2022}{}}%
Covatti Ale, M., Fischer, L., Deli Antoni, M., Diaz de Astarloa, J. M.,
\& Delpiani, G. (2022). Trophic ecology of the yellowfin notothen,
{Patagonotothen} guntheri ({Norman}, 1937) at the {Marine Protected Area
Namuncurá-Burdwood Bank}, {Argentina}. \emph{Polar Biology},
\emph{45}(4), 549--558. \url{https://doi.org/10.1007/s00300-022-03011-w}

\leavevmode\vadjust pre{\hypertarget{ref-Csardi2006}{}}%
Csardi, \& Nepusz. (2006). \emph{The igraph software package for complex
network research}.

\leavevmode\vadjust pre{\hypertarget{ref-Cury2000}{}}%
Cury, P., Bakun, A., Crawford, R. J. M., Jarre, A., Quiñones, R. A.,
Shannon, L. J., \& Verheye, H. M. (2000). Small pelagics in upwelling
systems: Patterns of interaction and structural changes in
{``wasp-waist''} ecosystems. \emph{ICES Journal of Marine Science},
\emph{57}(3), 603--618. \url{https://doi.org/10.1006/jmsc.2000.0712}

\leavevmode\vadjust pre{\hypertarget{ref-Day2013}{}}%
Day, R. H., Weingartner, T. J., Hopcroft, R. R., Aerts, L. A. M.,
Blanchard, A. L., Gall, A. E., Gallaway, B. J., Hannay, D. E., Holladay,
B. A., Mathis, J. T., Norcross, B. L., Questel, J. M., \& Wisdom, S. S.
(2013). The offshore northeastern {Chukchi Sea}, {Alaska}: {A} complex
high-latitude ecosystem. \emph{Continental Shelf Research}, \emph{67},
147--165. \url{https://doi.org/10.1016/j.csr.2013.02.002}

\leavevmode\vadjust pre{\hypertarget{ref-Dellabianca2023}{}}%
Dellabianca, N. A., Torres, M. A., Ordoñez, C., Fioramonti, N., \& Raya
Rey, A. (2023). Marine protected areas in the southern south-west
{Atlantic}: {Insights} from marine top predator communities.
\emph{Aquatic Conservation: Marine and Freshwater Ecosystems},
\emph{33}(5), 472--487. \url{https://doi.org/10.1002/aqc.3935}

\leavevmode\vadjust pre{\hypertarget{ref-Delpiani2020}{}}%
Delpiani, S. M., Bruno, D. O., Vazquez, D. M., Llompart, F., Delpiani,
G. E., Fernández, D. A., Rosso, J. J., Mabragaña, E., \& Díaz de
Astarloa, J. M. (2020). Structure and distribution of fish assemblages
at {Burdwood Bank}, the first {Sub-Antarctic Marine Protected Area}
{``{Namuncurá}''} in {Argentina} ({Southwestern Atlantic Ocean}).
\emph{Polar Biology}, \emph{43}(11), 1783--1793.
\url{https://doi.org/10.1007/s00300-020-02744-w}

\leavevmode\vadjust pre{\hypertarget{ref-DiMauro2022}{}}%
Di Mauro, R., Castillo, S., Pérez, A., Iachetti, C. M., Silva, L.,
Tomba, J. P., \& Chiesa, I. L. (2022). Anthropogenic microfibers are
highly abundant at the {Burdwood Bank} seamount, a protected
sub-{Antarctic} environment in the {Southwestern Atlantic Ocean}.
\emph{Environmental Pollution}, \emph{306}, 119364.
\url{https://doi.org/10.1016/j.envpol.2022.119364}

\leavevmode\vadjust pre{\hypertarget{ref-Dormann2017}{}}%
Dormann, C. F., Fründ, J., \& Schaefer, H. M. (2017). Identifying
{Causes} of {Patterns} in {Ecological Networks}: {Opportunities} and
{Limitations}. \emph{Annual Review of Ecology, Evolution, and
Systematics}, \emph{48}, 559--584.
\url{https://doi.org/10.1146/annurev-ecolsys-110316-022928}

\leavevmode\vadjust pre{\hypertarget{ref-Dunne2002}{}}%
Dunne, J. A., Williams, R. J., \& Martinez, N. D. (2002a). Network
structure and biodiversity loss in food webs: Robustness increases with
connectance. \emph{Ecology Letters}, \emph{5}(4), 558--567.
\url{https://doi.org/10.1046/j.1461-0248.2002.00354.x}

\leavevmode\vadjust pre{\hypertarget{ref-Dunne2002b}{}}%
Dunne, J. A., Williams, R. J., \& Martinez, N. D. (2002b). \emph{Small
{Networks} but not {Small Worlds}: {Unique Aspects} of {Food Web
Structure}}. {Santa Fe Institute}.

\leavevmode\vadjust pre{\hypertarget{ref-Dunne2002a}{}}%
Dunne, J. A., Williams, R. J., \& Martinez, N. D. (2002c). Food-web
structure and network theory: {The} role of connectance and size.
\emph{Proceedings of the National Academy of Sciences}, \emph{99}(20),
12917--12922. \url{https://doi.org/10.1073/pnas.192407699}

\leavevmode\vadjust pre{\hypertarget{ref-Eklof2006}{}}%
Eklöf, A., \& Ebenman, B. (2006). Species loss and secondary extinctions
in simple and complex model communities. \emph{Journal of Animal
Ecology}, \emph{75}(1), 239--246.
\url{https://doi.org/10.1111/j.1365-2656.2006.01041.x}

\leavevmode\vadjust pre{\hypertarget{ref-Fagan1997}{}}%
Fagan, W. F. (1997). Omnivory as a {Stabilizing Feature} of {Natural
Communities}. \emph{The American Naturalist}, \emph{150}(5), 554--567.
\url{https://doi.org/10.1086/286081}

\leavevmode\vadjust pre{\hypertarget{ref-Falabella2017}{}}%
Falabella, V. (2017). \emph{Área {Marina Protegida Namuncurá-Banco
Burdwood}. {Contribuciones} para la línea de base y el plan de manejo}.

\leavevmode\vadjust pre{\hypertarget{ref-Fioramonti2022}{}}%
Fioramonti, N. E., Ribeiro Guevara, S., Becker, Y. A., \& Riccialdelli,
L. (2022). Mercury transfer in coastal and oceanic food webs from the
{Southwest Atlantic Ocean}. \emph{Marine Pollution Bulletin},
\emph{175}, 113365.
\url{https://doi.org/10.1016/j.marpolbul.2022.113365}

\leavevmode\vadjust pre{\hypertarget{ref-Fischer2022}{}}%
Fischer, L., Covatti Ale, M., Deli Antoni, M., Díaz de Astarloa, J. M.,
\& Delpiani, G. (2022). Feeding ecology of the longtail southern cod,
{Patagonotothen} ramsayi ({Regan}, 1913) ({Notothenioidei}) in the
{Marine Protected Area Namuncurá-Burdwood Bank}, {Argentina}.
\emph{Polar Biology}, \emph{45}(9), 1483--1494.
\url{https://doi.org/10.1007/s00300-022-03082-9}

\leavevmode\vadjust pre{\hypertarget{ref-Florencia2023}{}}%
Florencia, M., Vazquez, D. M., Gabbanelli, V., Díaz de Astarloa, J. M.,
\& Mabragaña, E. (2023). Chondrichthyans from the southern tip of {South
America} with emphasis on the marine protected area {Namuncurá-Burdwood
Bank}: Exploring egg nursery grounds. \emph{Polar Biology}.
\url{https://doi.org/10.1007/s00300-023-03128-6}

\leavevmode\vadjust pre{\hypertarget{ref-Funes2022}{}}%
Funes, M., Saravia, L. A., Cordone, G., Iribarne, O. O., \& Galván, D.
E. (2022). Network analysis suggests changes in food web stability
produced by bottom trawl fishery in {Patagonia}. \emph{Scientific
Reports}, \emph{12}(1), 10876.
\url{https://doi.org/10.1038/s41598-022-14363-y}

\leavevmode\vadjust pre{\hypertarget{ref-GarciaAlonso2020}{}}%
García Alonso, V. A., Brown, D. R., Pájaro, M., \& Capitanio, F. L.
(2020). Growing {Up Down South}: {Spatial} and {Temporal Variability} in
{Early Growth} of {Fuegian Sprat Sprattus} fuegensis {From} the
{Southwest Atlantic Ocean}. \emph{Frontiers in Marine Science},
\emph{7}.

\leavevmode\vadjust pre{\hypertarget{ref-GarciaAlonso2018}{}}%
García Alonso, V. A., Brown, D., Martín, J., Pájaro, M., \& Capitanio,
F. L. (2018). Seasonal patterns of {Patagonian} sprat {Sprattus}
fuegensis early life stages in an open sea {Sub-Antarctic Marine
Protected Area}. \emph{Polar Biology}, \emph{41}(11), 2167--2179.
\url{https://doi.org/10.1007/s00300-018-2352-z}

\leavevmode\vadjust pre{\hypertarget{ref-Gauzens2013}{}}%
Gauzens, B., Legendre, S., Lazzaro, X., \& Lacroix, G. (2013). Food-web
aggregation, methodological and functional issues. \emph{Oikos},
\emph{122}(11), 1606--1615.
\url{https://doi.org/10.1111/j.1600-0706.2013.00266.x}

\leavevmode\vadjust pre{\hypertarget{ref-Giesecke2021}{}}%
Giesecke, R., Martín, J., Piñones, A., Höfer, J., Garcés-Vargas, J.,
Flores-Melo, X., Alarcón, E., Durrieu de Madron, X., Bourrin, F., \&
González, H. E. (2021). General {Hydrography} of the {Beagle Channel}, a
{Subantarctic Interoceanic Passage} at the {Southern Tip} of {South
America}. \emph{Frontiers in Marine Science}, \emph{8}.

\leavevmode\vadjust pre{\hypertarget{ref-Glorioso1995}{}}%
Glorioso, P. D., \& Flather, R. A. (1995). A barotropic model of the
currents off {SE South America}. \emph{Journal of Geophysical Research:
Oceans}, \emph{100}(C7), 13427--13440.
\url{https://doi.org/10.1029/95JC00942}

\leavevmode\vadjust pre{\hypertarget{ref-Godoy2018}{}}%
Godoy, O., Bartomeus, I., Rohr, R. P., \& Saavedra, S. (2018). Towards
the {Integration} of {Niche} and {Network Theories}. \emph{Trends in
Ecology \& Evolution}, \emph{33}(4), 287--300.
\url{https://doi.org/10.1016/j.tree.2018.01.007}

\leavevmode\vadjust pre{\hypertarget{ref-Guerrero1999}{}}%
Guerrero, R. A., Baldoni, A. G., \& Benavides, H. R. (1999).
\emph{Oceanographic conditions at the southern end of the argentine
continental slope}.
https://doi.org/\url{http://10.0.64.26/handle/inidep/247}

\leavevmode\vadjust pre{\hypertarget{ref-Guimera2005}{}}%
Guimerà, R., \& Nunes Amaral, L. A. (2005). Functional cartography of
complex metabolic networks. \emph{Nature}, \emph{433}(7028), 895--900.
\url{https://doi.org/10.1038/nature03288}

\leavevmode\vadjust pre{\hypertarget{ref-Guinder2020}{}}%
Guinder, V. A., Malits, A., Ferronato, C., Krock, B., Garzón-Cardona,
J., \& Martínez, A. (2020). Microbial plankton configuration in the
epipelagic realm from the {Beagle Channel} to the {Burdwood Bank}, a
{Marine Protected Area} in {Sub-Antarctic} waters. \emph{PLOS ONE},
\emph{15}(5), e0233156.
\url{https://doi.org/10.1371/journal.pone.0233156}

\leavevmode\vadjust pre{\hypertarget{ref-Heer2020}{}}%
Heer, H., Streib, L., Schäfer, R. B., \& Ruzika, S. (2020). Maximising
the clustering coefficient of networks and the effects on habitat
network robustness. \emph{PLOS ONE}, \emph{15}(10), e0240940.
\url{https://doi.org/10.1371/journal.pone.0240940}

\leavevmode\vadjust pre{\hypertarget{ref-Hogg2016}{}}%
Hogg, O. T., Huvenne, V. A. I., Griffiths, H. J., Dorschel, B., \&
Linse, K. (2016). Landscape mapping at sub-{Antarctic South Georgia}
provides a protocol for underpinning large-scale marine protected areas.
\emph{Scientific Reports}, \emph{6}(1), 33163.
\url{https://doi.org/10.1038/srep33163}

\leavevmode\vadjust pre{\hypertarget{ref-IUCN2023}{}}%
IUCN, \& UNEP-WCMC. (2023). \emph{The world database on protected areas
({WDPA})}. {Protected Planet}.

\leavevmode\vadjust pre{\hypertarget{ref-Kadoya2018}{}}%
Kadoya, T., Gellner, G., \& McCann, K. S. (2018). Potential oscillators
and keystone modules in food webs. \emph{Ecology Letters}, \emph{21}(9),
1330--1340. \url{https://doi.org/10.1111/ele.13099}

\leavevmode\vadjust pre{\hypertarget{ref-Kondoh2010}{}}%
Kondoh, M., Kato, S., \& Sakato, Y. (2010). Food webs are built up with
nested subwebs. \emph{Ecology}, \emph{91}(11), 3123--3130.
\url{https://doi.org/10.1890/09-2219.1}

\leavevmode\vadjust pre{\hypertarget{ref-Kortsch2019}{}}%
Kortsch, S., Primicerio, R., Aschan, M., Lind, S., Dolgov, A. V., \&
Planque, B. (2019). Food-web structure varies along environmental
gradients in a high-latitude marine ecosystem. \emph{Ecography},
\emph{42}(2), 295--308. \url{https://doi.org/10.1111/ecog.03443}

\leavevmode\vadjust pre{\hypertarget{ref-Kortsch2015}{}}%
Kortsch, S., Primicerio, R., Fossheim, M., Dolgov, A. V., \& Aschan, M.
(2015). Climate change alters the structure of arctic marine food webs
due to poleward shifts of boreal generalists. \emph{Proceedings of the
Royal Society B: Biological Sciences}, \emph{282}(1814), 20151546.
\url{https://doi.org/10.1098/rspb.2015.1546}

\leavevmode\vadjust pre{\hypertarget{ref-Lai2012}{}}%
Lai, S.-M., Liu, W.-C., \& Jordán, F. (2012). On the centrality and
uniqueness of species from the network perspective. \emph{Biology
Letters}, \emph{8}(4), 570--573.
\url{https://doi.org/10.1098/rsbl.2011.1167}

\leavevmode\vadjust pre{\hypertarget{ref-Laptikhovsky2013}{}}%
Laptikhovsky, V., Arkhipkin, A., \& Brickle, P. (2013). From small
bycatch to main commercial species: {Explosion} of stocks of rock cod
{Patagonotothen} ramsayi ({Regan}) in the {Southwest Atlantic}.
\emph{Fisheries Research}, \emph{147}, 399--403.
\url{https://doi.org/10.1016/j.fishres.2013.05.006}

\leavevmode\vadjust pre{\hypertarget{ref-Latorre2023}{}}%
Latorre, M. P., Berghoff, C. F., Giesecke, R., Malits, A., Pizarro, G.,
Iachetti, C. M., Martin, J., Flores-Melo, X., Gil, M. N., Iriarte, J.
L., \& Schloss, I. R. (2023). Plankton metabolic balance in the eastern
{Beagle Channel} during spring. \emph{Journal of Marine Systems},
\emph{240}, 103882. \url{https://doi.org/10.1016/j.jmarsys.2023.103882}

\leavevmode\vadjust pre{\hypertarget{ref-Lopez-Gappa2018}{}}%
López-Gappa, J., Liuzzi, M. G., \& Zelaya, D. G. (2018). A new genus and
species of cheilostome bryozoan associated with hermit crabs in the
subantarctic {Southwest Atlantic}. \emph{Polar Biology}, \emph{41}(4),
733--741. \url{https://doi.org/10.1007/s00300-017-2234-9}

\leavevmode\vadjust pre{\hypertarget{ref-Marina2018}{}}%
Marina, T. I., Salinas, V., Cordone, G., Campana, G., Moreira, E.,
Deregibus, D., Torre, L., Sahade, R., Tatián, M., Barrera Oro, E., De
Troch, M., Doyle, S., Quartino, M. L., Saravia, L. A., \& Momo, F. R.
(2018). The {Food Web} of {Potter Cove} ({Antarctica}): Complexity,
structure and function. \emph{Estuarine, Coastal and Shelf Science},
\emph{200}, 141--151. \url{https://doi.org/10.1016/j.ecss.2017.10.015}

\leavevmode\vadjust pre{\hypertarget{ref-Marina2018a}{}}%
Marina, T. I., Saravia, L. A., Cordone, G., Salinas, V., Doyle, S. R.,
\& Momo, F. R. (2018). Architecture of marine food webs: {To} be or not
be a {``small-world.''} \emph{PLOS ONE}, \emph{13}(5), e0198217.
\url{https://doi.org/10.1371/journal.pone.0198217}

\leavevmode\vadjust pre{\hypertarget{ref-Marina2023}{}}%
Marina, T. I., Saravia, L. A., \& Kortsch, S. (2023). \emph{New insights
into the {Weddell Sea} ecosystem applying a quantitative network
approach}. {EGUsphere preprint}.
\url{https://doi.org/10.5194/egusphere-2022-1518}

\leavevmode\vadjust pre{\hypertarget{ref-Martin2021}{}}%
Martin, J., \& Flores Melo, X. (2021). \emph{{Área Marina Protegida
Namuncurá Banco Burdwood: Aspectos físicos y biogeoquímicos}}. {Tercer
taller científico sobre el Área Marina Protegida Namuncurá Banco
Burdwood. Administración de Parques Nacionales (APN), Argentina}.

\leavevmode\vadjust pre{\hypertarget{ref-MartinSirito2019}{}}%
Martin Sirito, S. (2019). \emph{{Fauna asociada a corales (Octocorallia)
e hidroides (Hydrozoa) del Área Marina Protegida {``Namuncurá''} (Banco
Burdwood) y zonas profundas adyacentes}} {[}PhD thesis{]}. Universidad
Nacional de Mar del Plata (UNMdP), Argentina.

\leavevmode\vadjust pre{\hypertarget{ref-Martinez1991}{}}%
Martinez, N. D. (1991). Artifacts or {Attributes}? {Effects} of
{Resolution} on the {Little Rock Lake Food Web}. \emph{Ecological
Monographs}, \emph{61}(4), 367--392.
\url{https://doi.org/10.2307/2937047}

\leavevmode\vadjust pre{\hypertarget{ref-Martinez1993}{}}%
Martinez, N. D. (1993). Effects of {Resolution} on {Food Web Structure}.
\emph{Oikos}, \emph{66}(3), 403--412.
\url{https://doi.org/10.2307/3544934}

\leavevmode\vadjust pre{\hypertarget{ref-Martinez2021}{}}%
Martínez, P. A., Wöhler, O. C., Troccoli, G. H., \& Di Marco, E. J.
(2021). \emph{{Análisis del impacto potencial provocado por el
establecimiento de las áreas marinas protegidas Namuncurá-Banco Burdwood
I, II y Yaganes en la pesquería argentina de merluza negra (Dissostichus
eleginoides)}} (\{Technical report\} No. 23/2021; p. 17). {Instituto
Nacional de Investigación y Desarrollo Pesquero (INIDEP), Argentina}.

\leavevmode\vadjust pre{\hypertarget{ref-Martinez2015}{}}%
Martínez, P. A., Wöhler, O. G., \& Troccoli, G. H. (2015). \emph{{La
evolución de la pesquería de merluza negra (Dissostichus eleginoides) en
el espacio marítimo argentino. Periodo 2003- 2014}} (\{Technical
report\} No. 11/15; p. 12). {Instituto Nacional de Investigación y
Desarrollo Pesquero (INIDEP), Argentina}.

\leavevmode\vadjust pre{\hypertarget{ref-Matano2019}{}}%
Matano, R. P., Palma, E. D., \& Combes, V. (2019). The {Burdwood Bank
Circulation}. \emph{Journal of Geophysical Research: Oceans},
\emph{124}(10), 6904--6926. \url{https://doi.org/10.1029/2019JC015001}

\leavevmode\vadjust pre{\hypertarget{ref-Matusevich2022}{}}%
Matusevich. (2022). \emph{Chondrichthyan fauna from the {Marine
Protected Area Namuncurá} at {Burdwood Bank}: Exploring egg nursery
grounds}. https://www.researchsquare.com.
\url{https://doi.org/10.21203/rs.3.rs-2247873/v1}

\leavevmode\vadjust pre{\hypertarget{ref-McCann1997}{}}%
McCann, K., \& Hastings, A. (1997). Re\textendash evaluating the
omnivory\textendash stability relationship in food webs.
\emph{Proceedings of the Royal Society of London. Series B: Biological
Sciences}, \emph{264}(1385), 1249--1254.
\url{https://doi.org/10.1098/rspb.1997.0172}

\leavevmode\vadjust pre{\hypertarget{ref-Moles2015}{}}%
Moles, J., Núñez-Pons, L., Taboada, S., Figuerola, B., Cristobo, J., \&
Avila, C. (2015). Anti-predatory chemical defences in {Antarctic}
benthic fauna. \emph{Marine Biology}, \emph{162}(9), 1813--1821.
\url{https://doi.org/10.1007/s00227-015-2714-9}

\leavevmode\vadjust pre{\hypertarget{ref-Montoya2002}{}}%
Montoya, J. M., \& Solé, R. V. (2002). Small {World Patterns} in {Food
Webs}. \emph{Journal of Theoretical Biology}, \emph{214}(3), 405--412.
\url{https://doi.org/10.1006/jtbi.2001.2460}

\leavevmode\vadjust pre{\hypertarget{ref-Montoya2003}{}}%
Montoya, J. M., \& Solé, R. V. (2003). Topological properties of food
webs: From real data to community assembly models. \emph{Oikos},
\emph{102}(3), 614--622.
\url{https://doi.org/10.1034/j.1600-0706.2003.12031.x}

\leavevmode\vadjust pre{\hypertarget{ref-Morales-Castilla2015}{}}%
Morales-Castilla, I., Matias, M. G., Gravel, D., \& Araújo, M. B.
(2015). Inferring biotic interactions from proxies. \emph{Trends in
Ecology \& Evolution}, \emph{30}(6), 347--356.
\url{https://doi.org/10.1016/j.tree.2015.03.014}

\leavevmode\vadjust pre{\hypertarget{ref-Neutel2002}{}}%
Neutel, A.-M., Heesterbeek, J. A. P., \& de Ruiter, P. C. (2002).
Stability in {Real Food Webs}: {Weak Links} in {Long Loops}.
\emph{Science}, \emph{296}(5570), 1120--1123.
\url{https://doi.org/10.1126/science.1068326}

\leavevmode\vadjust pre{\hypertarget{ref-Newman2003}{}}%
Newman, M. E. J. (2003). The {Structure} and {Function} of {Complex
Networks}. \emph{SIAM Review}, \emph{45}(2), 167--256.
\url{https://doi.org/10.1137/S003614450342480}

\leavevmode\vadjust pre{\hypertarget{ref-Nilsson2016}{}}%
Nilsson, K. A., \& McCann, K. S. (2016). Interaction strength
revisited\textemdash clarifying the role of energy flux for food web
stability. \emph{Theoretical Ecology}, \emph{9}(1), 59--71.
\url{https://doi.org/10.1007/s12080-015-0282-8}

\leavevmode\vadjust pre{\hypertarget{ref-Nunez-Pons2010}{}}%
Núñez-Pons, L., Forestieri, R., Nieto, R. M., Varela, M., Nappo, M.,
Rodríguez, J., Jiménez, C., Castelluccio, F., Carbone, M., Ramos-Espla,
A., Gavagnin, M., \& Avila, C. (2010). Chemical defenses of tunicates of
the genus {Aplidium} from the {Weddell Sea} ({Antarctica}). \emph{Polar
Biology}, \emph{33}(10), 1319--1329.
\url{https://doi.org/10.1007/s00300-010-0819-7}

\leavevmode\vadjust pre{\hypertarget{ref-Padovani2021}{}}%
Padovani, L. N., Álvarez, N., \& Farías, A. (2021). \emph{Alimentación
de la sardina fueguina ({Sprattus} fuegensis) en la región patagónica
austral durante la época reproductiva} {[}Technical Report{]}.
{Instituto Nacional de Investigación y Desarrollo Pesquero (INIDEP)}.

\leavevmode\vadjust pre{\hypertarget{ref-Padovani2012}{}}%
Padovani, L. N., Viñas, M. D., Sánchez, F., \& Mianzan, H. (2012).
Amphipod-supported food web: {Themisto} gaudichaudii, a key food
resource for fishes in the southern {Patagonian Shelf}. \emph{Journal of
Sea Research}, \emph{67}(1), 85--90.
\url{https://doi.org/10.1016/j.seares.2011.10.007}

\leavevmode\vadjust pre{\hypertarget{ref-Pascual2005}{}}%
Pascual, M., \& Dunne, J. A. (2005). \emph{Ecological {Networks}:
{Linking Structure} to {Dynamics} in {Food Webs}}. {Oxford University
Press}.

\leavevmode\vadjust pre{\hypertarget{ref-Pawar2012}{}}%
Pawar, S., Dell, A. I., \& Van M. Savage. (2012). Dimensionality of
consumer search space drives trophic interaction strengths.
\emph{Nature}, \emph{486}, 485.
\url{https://doi.org/10.1038/nature11131}

\leavevmode\vadjust pre{\hypertarget{ref-Piola2009}{}}%
Piola, A. R., \& Falabella, V. (2009). {El mar Patagónico}. In
\emph{{Atlas del Mar Patagónico. Especies y Espacios}} (Falabella, V.,
Campagna, C. y Croxall, J. Ed., pp. 54--75). {Wildlife Conservation
Society y Birdlife Internacional}.

\leavevmode\vadjust pre{\hypertarget{ref-Piola1989}{}}%
Piola, A. R., \& Gordon, A. L. (1989). Intermediate waters in the
southwest {South Atlantic}. \emph{Deep Sea Research Part A.
Oceanographic Research Papers}, \emph{36}(1), 1--16.
\url{https://doi.org/10.1016/0198-0149(89)90015-0}

\leavevmode\vadjust pre{\hypertarget{ref-Poisot2014}{}}%
Poisot, T., \& Gravel, D. (2014). When is an ecological network complex?
{Connectance} drives degree distribution and emerging network
properties. \emph{PeerJ}, \emph{2}, e251.
\url{https://doi.org/10.7717/peerj.251}

\leavevmode\vadjust pre{\hypertarget{ref-Pomeranz2019}{}}%
Pomeranz, J. P. F., Thompson, R. M., Poisot, T., \& Harding, J. S.
(2019). Inferring predator\textendash prey interactions in food webs.
\emph{Methods in Ecology and Evolution}, \emph{10}(3), 356--367.
\url{https://doi.org/10.1111/2041-210X.13125}

\leavevmode\vadjust pre{\hypertarget{ref-Presta2023}{}}%
Presta, M. L., Riccialdelli, L., Bruno, D. O., Castro, L. R.,
Fioramonti, N. E., Florentín, O. V., Berghoff, C. F., Capitanio, F. L.,
\& Lovrich, G. A. (2023). Mesozooplankton community structure and
trophic relationships in an austral high-latitude ecosystem ({Beagle
Channel}): {The} role of bottom-up and top-down forces during
springtime. \emph{Journal of Marine Systems}, \emph{240}, 103881.
\url{https://doi.org/10.1016/j.jmarsys.2023.103881}

\leavevmode\vadjust pre{\hypertarget{ref-Prieto2022}{}}%
Prieto, I. M., Paola, A., Pérez, M., García, M., Blustein, G., Schejter,
L., \& Palermo, J. A. (2022). Antifouling {Diterpenoids} from the
{Sponge} {\emph{Dendrilla}}{ \emph{Antarctica}}. \emph{Chemistry \&
Biodiversity}, \emph{19}(2).
\url{https://doi.org/10.1002/cbdv.202100618}

\leavevmode\vadjust pre{\hypertarget{ref-Reta2014}{}}%
Reta, R. (2014). \emph{Oceanografía del {Banco Burdwood}: {Estado
Actual} del {Conocimiento} y {Perspectivas}}. {Instituto Nacional de
Investigación y Desarrollo Pesquero (INIDEP), Argentina}.

\leavevmode\vadjust pre{\hypertarget{ref-Riccialdelli2020}{}}%
Riccialdelli, L., Becker, Y. A., Fioramonti, N. E., Torres, M., Bruno,
D. O., Rey, A. R., \& Fernández, D. A. (2020). Trophic structure of
southern marine ecosystems: A comparative isotopic analysis from the
{Beagle Channel} to the oceanic {Burdwood Bank} area under a wasp-waist
assumption. \emph{Marine Ecology Progress Series}, \emph{655}, 1--27.
\url{https://doi.org/10.3354/meps13524}

\leavevmode\vadjust pre{\hypertarget{ref-Roberts2017}{}}%
Roberts, C. M., O'Leary, B. C., McCauley, D. J., Cury, P. M., Duarte, C.
M., Lubchenco, J., Pauly, D., Sáenz-Arroyo, A., Sumaila, U. R., Wilson,
R. W., Worm, B., \& Castilla, J. C. (2017). Marine reserves can mitigate
and promote adaptation to climate change. \emph{Proceedings of the
National Academy of Sciences}, \emph{114}(24), 6167--6175.
\url{https://doi.org/10.1073/pnas.1701262114}

\leavevmode\vadjust pre{\hypertarget{ref-Rodriguez2022}{}}%
Rodriguez, I. D., Marina, T. I., Schloss, I. R., \& Saravia, L. A.
(2022). Marine food webs are more complex but less stable in
sub-{Antarctic} ({Beagle Channel}, {Argentina}) than in {Antarctic}
({Potter Cove}, {Antarctic Peninsula}) regions. \emph{Marine
Environmental Research}, \emph{174}, 105561.
\url{https://doi.org/10.1016/j.marenvres.2022.105561}

\leavevmode\vadjust pre{\hypertarget{ref-RojodeAlmeida2010}{}}%
Rojo de Almeida, M. T., Siless, G. E., Perez, C. D., Veloso, M. J.,
Schejter, L., Puricelli, L., \& Palermo, J. A. (2010). Dolabellane
{Diterpenoids} from the {South Atlantic Gorgonian} {\emph{Convexella}}{
\emph{Magelhaenica}}. \emph{Journal of Natural Products}, \emph{73}(10),
1714--1717. \url{https://doi.org/10.1021/np100337j}

\leavevmode\vadjust pre{\hypertarget{ref-Sala2018}{}}%
Sala, E., Lubchenco, J., Grorud-Colvert, K., Novelli, C., Roberts, C.,
\& Sumaila, U. R. (2018). Assessing real progress towards effective
ocean protection. \emph{Marine Policy}, \emph{91}, 11--13.
\url{https://doi.org/10.1016/j.marpol.2018.02.004}

\leavevmode\vadjust pre{\hypertarget{ref-Santana2013}{}}%
Santana, C. N. de, Rozenfeld, A. F., Marquet, P. A., \& Duarte, C. M.
(2013). Topological properties of polar food webs. \emph{Marine Ecology
Progress Series}, \emph{474}, 15--26.
\url{https://doi.org/10.3354/meps10073}

\leavevmode\vadjust pre{\hypertarget{ref-Saporiti2015}{}}%
Saporiti, F., Bearhop, S., Vales, D. G., Silva, L., Zenteno, L.,
Tavares, M., Crespo, E. A., \& Cardona, L. (2015). Latitudinal changes
in the structure of marine food webs in the {Southwestern Atlantic
Ocean}. \emph{Marine Ecology Progress Series}, \emph{538}, 23--34.
\url{https://doi.org/10.3354/meps11464}

\leavevmode\vadjust pre{\hypertarget{ref-Saravia2022a}{}}%
Saravia, L. A. (2022). \emph{Multiweb: {Ecological} network analyses
including multiplex networks}.

\leavevmode\vadjust pre{\hypertarget{ref-Schejter2021}{}}%
Schejter, L., \& Albano, M. (2021). Benthic communities at the marine
protected area {Namuncurá}/{Burdwood} bank, {SW Atlantic Ocean}:
Detection of vulnerable marine ecosystems and contributions to the
assessment of the rezoning process. \emph{Polar Biology}, \emph{44}(10),
2023--2037. \url{https://doi.org/10.1007/s00300-021-02936-y}

\leavevmode\vadjust pre{\hypertarget{ref-Schejter2019}{}}%
Schejter, L., \& Bremec, C. S. (2019). Stony corals ({Anthozoa}:
{Scleractinia}) of {Burdwood Bank} and neighbouring areas, {SW Atlantic
Ocean}. \emph{Scientia Marina}, \emph{83}(3), 247--260.
\url{https://doi.org/10.3989/scimar.04863.10A}

\leavevmode\vadjust pre{\hypertarget{ref-Schejter2020}{}}%
Schejter, L., Genzano, G., Gaitán, E., Perez, C. D., \& Bremec, C. S.
(2020). Benthic communities in the {Southwest Atlantic Ocean}:
{Conservation} value of animal forests at the {Burdwood Bank} slope.
\emph{Aquatic Conservation: Marine and Freshwater Ecosystems},
\emph{30}(3), 426--439. \url{https://doi.org/10.1002/aqc.3265}

\leavevmode\vadjust pre{\hypertarget{ref-Schejter2016}{}}%
Schejter, L., Rimondino, C., Chiesa, I., Díaz de Astarloa, J. M., Doti,
B., Elías, R., Escolar, M., Genzano, G., López-Gappa, J., Tatián, M.,
Zelaya, D. G., Cristobo, J., Perez, C. D., Cordeiro, R. T., \& Bremec,
C. S. (2016). Namuncurá {Marine Protected Area}: An oceanic hot spot of
benthic biodiversity at {Burdwood Bank}, {Argentina}. \emph{Polar
Biology}, \emph{39}(12), 2373--2386.
\url{https://doi.org/10.1007/s00300-016-1913-2}

\leavevmode\vadjust pre{\hypertarget{ref-Scotti2010}{}}%
Scotti, M., \& Jordán, F. (2010). Relationships between centrality
indices and trophic levels in food webs. \emph{Community Ecology},
\emph{11}(1), 59--67. \url{https://doi.org/10.1556/ComEc.11.2010.1.9}

\leavevmode\vadjust pre{\hypertarget{ref-Secretariatoftheconventiononbiologicaldiversity2004}{}}%
Secretariat of the convention on biological diversity. (2004).
\emph{Technical advice on the establishment and management of a national
system of marine and coastal protected areas} (Technical Report No. 13).

\leavevmode\vadjust pre{\hypertarget{ref-Smith-Ramesh2017}{}}%
Smith-Ramesh, L. M., Moore, A. C., \& Schmitz, O. J. (2017). Global
synthesis suggests that food web connectance correlates to invasion
resistance. \emph{Global Change Biology}, \emph{23}(2), 465--473.
\url{https://doi.org/10.1111/gcb.13460}

\leavevmode\vadjust pre{\hypertarget{ref-Sole2001}{}}%
Solé, R. V., \& Montoya, M. (2001). Complexity and fragility in
ecological networks. \emph{Proceedings of the Royal Society of London.
Series B: Biological Sciences}, \emph{268}(1480), 2039--2045.
\url{https://doi.org/10.1098/rspb.2001.1767}

\leavevmode\vadjust pre{\hypertarget{ref-Spinelli2020}{}}%
Spinelli, M. L., Malits, A., García Alonso, V. A., Martín, J., \&
Capitanio, F. L. (2020). Spatial gradients of spring zooplankton
assemblages at the open ocean sub-{Antarctic Namuncurá Marine Protected
Area}/{Burdwood Bank}, {SW Atlantic Ocean}. \emph{Journal of Marine
Systems}, \emph{210}, 103398.
\url{https://doi.org/10.1016/j.jmarsys.2020.103398}

\leavevmode\vadjust pre{\hypertarget{ref-Stouffer2010}{}}%
Stouffer, D. B., \& Bascompte, J. (2010). Understanding food-web
persistence from local to global scales. \emph{Ecology Letters},
\emph{13}(2), 154--161.
\url{https://doi.org/10.1111/j.1461-0248.2009.01407.x}

\leavevmode\vadjust pre{\hypertarget{ref-Stouffer2005}{}}%
Stouffer, D. B., Camacho, J., Guimerà, R., Ng, C. A., \& Nunes Amaral,
L. A. (2005). Quantitative {Patterns} in the {Structure} of {Model} and
{Empirical Food Webs}. \emph{Ecology}, \emph{86}(5), 1301--1311.
\url{https://doi.org/10.1890/04-0957}

\leavevmode\vadjust pre{\hypertarget{ref-RCoreTeam2022}{}}%
Team, R. C. (2022). \emph{R: {A Language} and {Environment} for
{Statistical Computing}}. R Foundation for Statistical Computing.

\leavevmode\vadjust pre{\hypertarget{ref-Thebault2010}{}}%
Thébault, E., \& Fontaine, C. (2010). Stability of {Ecological
Communities} and the {Architecture} of {Mutualistic} and {Trophic
Networks}. \emph{Science}, \emph{329}(5993), 853--856.
\url{https://doi.org/10.1126/science.1188321}

\leavevmode\vadjust pre{\hypertarget{ref-Tombesi2020}{}}%
Tombesi, M. L., Rabufetti, F., \& Lovrich, G. A. (2020). Las áreas
marinas protegidas en la {Argentina}. \emph{La Lupa. Colección Fueguina
De Divulgación Científica}, \emph{16}, 2--7.
https://doi.org/\url{https://www.coleccionlalupa.com.ar/index.php/lalupa/article/view/79}

\leavevmode\vadjust pre{\hypertarget{ref-Trathan2021}{}}%
Trathan, P. N., Fielding, S., Hollyman, P. R., Murphy, E. J.,
Warwick-Evans, V., \& Collins, M. A. (2021). Enhancing the ecosystem
approach for the fishery for {Antarctic} krill within the complex,
variable, and changing ecosystem at {South Georgia}. \emph{ICES Journal
of Marine Science}, \emph{78}(6), 2065--2081.
\url{https://doi.org/10.1093/icesjms/fsab092}

\leavevmode\vadjust pre{\hypertarget{ref-Troccoli2020}{}}%
Troccoli, G. H., Aguilar, E., Martínez, P. A., \& Belleggia, M. (2020).
The diet of the {Patagonian} toothfish {Dissostichus} eleginoides, a
deep-sea top predator off {Southwest Atlantic Ocean}. \emph{Polar
Biology}, \emph{43}(10), 1595--1604.
\url{https://doi.org/10.1007/s00300-020-02730-2}

\leavevmode\vadjust pre{\hypertarget{ref-Valenzuela2018}{}}%
Valenzuela, L. O., Rowntree, V. J., Sironi, M., \& Seger, J. (2018).
Stable isotopes ({\(\delta\)15N}, {\(\delta\)13C}, {\(\delta\)34S}) in
skin reveal diverse food sources used by southern right whales
{Eubalaena} australis. \emph{Marine Ecology Progress Series},
\emph{603}, 243--255. \url{https://doi.org/10.3354/meps12722}

\leavevmode\vadjust pre{\hypertarget{ref-Vazquez2018}{}}%
Vazquez, D. M., Belleggia, M., Schejter, L., \& Mabragaña, E. (2018).
Avoiding being dragged away: Finding egg cases of {Schroederichthys}
bivius ({Chondrichthyes}: {Scyliorhinidae}) associated with benthic
invertebrates. \emph{Journal of Fish Biology}, \emph{92}(1), 248--253.
\url{https://doi.org/10.1111/jfb.13490}

\leavevmode\vadjust pre{\hypertarget{ref-Watts1998}{}}%
Watts, D. J., \& Strogatz, S. H. (1998). Collective dynamics of
{``small-world''} networks. \emph{Nature}, \emph{393}(6684), 440--442.
\url{https://doi.org/10.1038/30918}

\leavevmode\vadjust pre{\hypertarget{ref-Williams2002}{}}%
Williams, R. J., Berlow, E. L., Dunne, J. A., Barabási, A.-L., \&
Martinez, N. D. (2002). Two degrees of separation in complex food webs.
\emph{Proceedings of the National Academy of Sciences}, \emph{99}(20),
12913--12916. \url{https://doi.org/10.1073/pnas.192448799}

\leavevmode\vadjust pre{\hypertarget{ref-Winter2023}{}}%
Winter, A., \& Arkhipkin, A. (2023). Opportunistic {Survey Analyses
Reveal} a {Recent Decline} of {Skate} ({Rajiformes}) {Biomass} in
{Falkland Islands Waters}. \emph{Fishes}, \emph{8}(1), 24.
\url{https://doi.org/10.3390/fishes8010024}

\end{CSLReferences}


\end{document}
